% LaTeX Template for a Letter or Correspondence
% 
% To use:
%
% Copy into a new file, replace all
% [BRACKETED UPPER CASE TEXT]
% with your own, then run the latex command on it.
% Use dvips to print the .dvi output
\documentclass[11pt]{letter}
\usepackage[utf8]{inputenc}
\usepackage{times}
\usepackage[unicode=true,colorlinks=true,urlcolor=blue,citecolor=black,linkcolor=black]{hyperref}
\usepackage{geometry}
\geometry{verbose,tmargin= 1.5in,bmargin= 1.5in,
          lmargin= 1in,rmargin= 1in}
% If the letter doesn't fit on one page, but you
% want it do, remove the ``%'' on the next line:
% \setlength{\textheight}{8in}

\usepackage{enumitem}
\setlist{parsep=0.5em,itemsep=1em}

\begin{document}
\raggedright{}

% If you want headings on subsequent pages,
% remove the ``%'' on the next line:
% \pagestyle{headings}

\begin{letter}{ \\

}
\address{Integrative Research Center\\
  The Field Museum\\
  1400 South Lake Shore Drive\\
  Chicago, IL  60605-2496\\
  USA}

\opening{To the Editor:}

Thank you for handling our revised submission. As suggested by
Reviewer \#1, we have made further changes to (1) more clearly
integrate the supplementary analysis that addresses the treatment of
the Himalayas and QTP as a single biogeographic region, and (2) more
comprehensively refer to the new Figure 1 (map of the focal regions).

\textbf{Reviewer \#1:}

\textit{The only significant issue that still needs to be addressed is
  that the new analyses added treating the Himalaya and the QTP as
  separate areas have not been properly described and integrated into
  the manuscript in all the places where this is relevant.}

\begin{quote}
  Thank you for noticing this; we certainly agree and apologize for
  this inattention to detail.
\end{quote}

\textit{ In lines 475-479 at the end of the discussion these
  additional analyses are mentioned in relation to area
  delimitation. However, these new analyses should also surely be
  mentioned in: (i) para 2, page 3, lines 261-263 where the masking
  effects of treating the older QTP and younger Himalaya as a single
  region are mentioned, whereas this possible effect has now been
  tested?}

\begin{quote}
  We have rewritten this paragraph (line 257) to clarify that
  uplift-driven diversification in the Himalayas is not evident,
  regardless of it being lumped with the QTP or not.
\end{quote}

\textit{(ii) Materials and Methods, para3, lines 673-677 where the
  issue of treating the QTP and Himalaya as a single area is mentioned
  and justified because available data on species ranges often lacked
  sufficient detail to differentiate between the Himalaya and the
  QTP. How does this tally and reconcile with the additional analysis
  treating the QTP and Himalaya as separate areas?}

\begin{quote}
  We have rewritten the first 2 paragraphs of the section
  ``Delimitation of biogeographic regions'' (line 661) to clarify our
  rationale in focusing on the original analysis, in which the
  Himalayas and QTP are combined, instead of the one treating them
  separately. It boils down to confidence: for the latter analysis we
  did our best to score species ranges using the available
  information, but the process involved some guesswork. Moreover, the
  original analysis is arguably simpler and more cleanly shows the
  pattern of Hengduan assembly dynamics.

  \vspace{6pt}

  We have also inserted an introductory reference to this issue in the
  first paragraph of Results, line 153, since this will precede the
  Materials and Methods.
\end{quote}

\textit{The new Figure 1 is only sparsely referred to in the text and
  should be cross-referenced in paras 2 and 3 on page 1 and in the
  last paragraph of the discussion where the possible limitations of
  these area delimitations are mentioned.}

\begin{quote}
  References to Figure 1 are now in the Introduction (lines 62, 90,
  137) and Discussion (line 473).
\end{quote}

\textit{Methods, line 674, better: ... Our decision to treat the
  Himalaya and QTP as a single unit for ...}

\begin{quote}
  We have rewritten this paragraph (see line 669) and removed this
  phrasing.
\end{quote}


Thank you again handling our manuscript.

%\signature{Richard Ree}

\closing{Sincerely,\\
  \bigskip
  Richard Ree\\
  +1 312-665-7857\\
  rree@fieldmuseum.org\\
  \medskip on behalf of co-author Yaowu Xing
  }

% \encl{[ENCLOSURE LISTING]}

\end{letter}
\end{document}


%%% Local Variables: 
%%% mode: tex-pdf
%%% TeX-master: t
%%% End: 
