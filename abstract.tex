\section*{Abstract}

A common hypothesis for the rich biodiversity found in mountains is
``uplift-driven diversification''---that orogeny creates conditions
favoring rapid \textit{in situ} speciation of resident lineages. We
tested this hypothesis in the context of the Qinghai-Tibetan Plateau
(QTP) and adjoining mountain ranges using the phylogenetic and
geographic histories of multiple groups of plants to infer the tempo
(rate) and mode (colonization vs.\ \textit{in situ} diversification)
of biotic assembly through time and across regions. We focused on the
Hengduan Mountains region, which in comparison to the QTP and
Himalayas was uplifted more recently (since the late Miocene), is
smaller in area, and richer in species. The time-calibrated
phylogenetic analyses, which made no prior assumptions about when any
region was uplifted, show that after about 8 Ma, the rate of
\textit{in situ} diversification increased in the Hengduan Mountains,
significantly exceeding that of the geologically older QTP and
Himalayas, and yielded a remarkable inflection point at which
cumulative speciation overtakes colonization. By contrast, in the same
period for the QTP and Himalayas, the rate of \textit{in situ}
diversification remained relatively flat and lower than that of
colonization. This indicates that the Hengduan Mountains flora has
been assembled disproportionately by recent \textit{in situ}
diversification that coincides temporally with independent estimates
of orogeny, and is the first quantitative evidence to support the
uplift-driven diversification hypothesis.

%%% Local Variables: 
%%% mode: latex
%%% TeX-master: "master"
%%% End: 
