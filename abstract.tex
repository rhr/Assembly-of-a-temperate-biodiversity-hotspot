\section*{Abstract}

A common hypothesis for the rich biodiversity found in mountains is
uplift-driven diversification---that orogeny creates conditions
favoring rapid \textit{in situ} speciation of resident lineages. We
tested this hypothesis in the context of the Qinghai-Tibetan Plateau
(QTP) and adjoining mountain ranges, using the phylogenetic and
geographic histories of multiple groups of plants to infer the tempo
(rate) and mode (colonization vs.\ \textit{in situ} diversification)
of biotic assembly through time and across regions. We focused on the
Hengduan Mountains region, which in comparison to the QTP and
Himalayas was uplifted more recently (since the late Miocene), is
smaller in area, and richer in species. The time-calibrated
phylogenetic analyses, which made no prior assumptions about when any
region was uplifted, showed that about 8 million years ago, the rate
of \textit{in situ} diversification increased in the Hengduan
Mountains, significantly exceeding that in the geologically older QTP
and Himalayas, and marked the point at which cumulative speciation
overtook colonization. By contrast, in the QTP and Himalayas during
the same period, the rate of \textit{in situ} diversification remained
relatively flat, with colonization dominating lineage
accumulation. This indicates that the Hengduan Mountains flora has
been assembled disproportionately by recent \textit{in situ}
diversification that coincides temporally with independent estimates
of orogeny, and is the first quantitative evidence to support the
uplift-driven diversification hypothesis.

\newpage

\section*{Significance Statement}

Why do so many species occur in mountains? A popular but little-tested
hypothesis is that mountain uplift creates environmental conditions
(new habitats, dispersal barriers, etc.) that increase the rate at
which resident species divide and evolve to form new ones. In China's
Hengduan Mountains region, a biodiversity hotspot where uplift began
about 8 million years ago, this rate does in fact show a significant
increase since that time, relative to the rate for adjacent older
mountains, and to the rate of species immigration. The Hengduan
Mountains flora is thus made up disproportionately of species that
evolved within the region during its uplift, supporting the original
hypothesis and helping to explain the prevalence of mountains in
global biodiversity hotspots.



%%% Local Variables: 
%%% mode: latex
%%% TeX-master: "master"
%%% End: 
