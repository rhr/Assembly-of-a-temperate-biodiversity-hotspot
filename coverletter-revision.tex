
% LaTeX Template for a Letter or Correspondence
% 
% To use:
%
% Copy into a new file, replace all
% [BRACKETED UPPER CASE TEXT]
% with your own, then run the latex command on it.
% Use dvips to print the .dvi output
\documentclass[12pt]{letter}
\usepackage[utf8]{inputenc}
\usepackage{times}
\usepackage[unicode=true,colorlinks=true,urlcolor=blue,citecolor=black,linkcolor=black]{hyperref}
\usepackage{geometry}
\geometry{verbose,tmargin= 1.5in,bmargin= 1.5in,
          lmargin= 1in,rmargin= 1in}
% If the letter doesn't fit on one page, but you
% want it do, remove the ``%'' on the next line:
% \setlength{\textheight}{8in}

\usepackage{enumitem}
\setlist{parsep=1em}

\begin{document}
\raggedright{}

% If you want headings on subsequent pages,
% remove the ``%'' on the next line:
% \pagestyle{headings}

\begin{letter}{ \\

}
\address{Integrative Research Center\\
  The Field Museum\\
  1400 South Lake Shore Drive\\
  Chicago, IL  60605-2496\\
  USA}

\opening{To the Editor:}

Thank you for the constructive reviews. To address them, we have added
new data, performed new analyses, and revised the manuscript, as
detailed below.


\textbf{Editor comments:}

\textit{It will be important to address the numerous comments of
  Reviewer \#1, particularly regarding the wider context of the study,
  and to somehow address the BAMM issue. I am not suggesting you
  remove the BAMM analyses but you do need to address the recent
  critique.}

\textbf{Reviewer \#1:}

\begin{enumerate}
\item \textit{P3, column 1, para 2. While the authors justify
    delimitation of the Hengduan mountains as a region separate from
    the rest of the Himalaya and the QTP on the basis of differences
    in the timing of uplift, this needs to be qualified somewhat. The
    eastern Himalaya, QTP and Hengduan adjoin directly and the
    boundary between these regions is not clear cut. The uplift
    histories along these boundaries would clearly have been
    connected, complex and to some extent contiguous, suggesting that
    these area delimitations involve at least an arbitrary element as
    to where the boundaries are drawn. I think this should be
    acknowledged, and the possible sensitivity of the results to where
    this boundary is placed discussed in the paper.}

  We have added a paragraph to the end of the Discussion that
  addresses this issue.

\item \textit{P3, column 2, para 2. Is it really precise enough to
    describe the possible greater Pleistocene survival in the Hengduan
    cf the QTP in terms of refugia? Refugia are often taken to imply
    geographically displaced and geographically isolated areas
    surrounding ice caps. It seems entirely possible that in the
    Hengduan mountains a rather different pattern of simple downward
    altitudinal displacement onto geographically adjacent lower
    slopes, rather than larger scale geographical displacement during
    glacial cycles would have predominated. This could potentially
    have resulted in greater connectivity of alpine habitats at
    glacial maxima, and might provide a more precise explanation for
    lower extinction in the Hengduan than the QTP.}

  
\item \textit{P4, column 1, lines 394-398. Yes, a strong role for
    Pleistocene oscillations in diversification of the Paramo flora in
    the Andes seems entirely justified. However, most authors have
    also emphasized recent uplift history and the availability of new
    tropicalpine habitats (i.e. ecological opportunity) as the main
    driver of rapid recent diversification in the high elevation
    Andean grasslands. What seems to stand out from this study of
    Hengduan diversification cf the high elevation Andes, is the much
    greater role for colonization in the assembly of the Hengduan
    clades than that found so far in the Andes.  Current evidence
    suggests a much more limited role for colonization in the high
    elevation Andes with a predominance of in situ diversification, in
    that many of the large Andean radiations are monophyletic or
    biphyletic, reflecting the much greater geographic isolation of
    the Andes from similar cool habitats cf the Hengduan. New Zealand
    alpine habitats seem to follow the Andes in this respect with in
    situ diversification dominating.  Thus, although in situ
    diversification is dominant in the later phases of diversification
    in the Hengduan, rates of colonization remain much higher than for
    most high elevation Andean groups. I believe that including and
    highlighting this apparently significant difference will enhance
    the general interest in this paper.}

\item \textit{P4, column 1, lines 401-404. Related to the above, this
    point about possible differences in diversification trajectories
    between different habitats in the Hengduan seems to have been
    rather glossed over in this study. The study clades used here
    encompass both forest and higher elevation alpine habitats. Some
    clades (Pinaceae, Acer etc.)  are restricted to the lower
    elevation forest zone, while others (e.g. Saussurea etc.) are
    restricted to higher elevation alpine habitats. It seems very
    likely that patterns of diversification (timing, rates, balance
    between colonization and in situ diversification) could have been
    rather different in these different habitats.  Notably, this
    appears to be very much the case in the Andes, where inter-Andean
    valley, mid elevation montane forest and high altitude grassland
    clades show very different patterns in terms of timing and rates
    of species diversification (see Sarkinen et al. 2012 J Biogeog 39:
    884-900; Hughes 2016 The tropical Andean plant diversity
    powerhouse. New Phytologist 210: 1152-1154). By lumping all
    Hengduan clades from diverse habitats together in the analyses
    presented here seems like an over-simplification. Do the forest
    clades differ from the alpine clades? This should be tested. Do
    the patterns presented here potentially apply only to alpine
    clades?}

\item \textit{P4, column 1, lines 399-401. These statements about the
    Cerrado are over-simplifications.}

  \textit{First, the idea of "colonization" with its implication of
    some form of geographical dispersal from adjacent areas, may not
    apply in the case of the Cerrado, where more likely Cerrado
    lineages simply adapted from ancestral forest lineages to tolerate
    fire in situ with the expansion of savanna grasslands associated
    with Pliocene formation of the Cerrado, as suggested by Simon et
    al (2009).}

  \textit{Furthermore, it is also wrong to suggest that there is
    limited in situ diversification of Cerrado lineages. There are
    several well-documented examples of clades in the Cerrado that
    show accelerated rates of in situ species diversification
    generating clades of up to 50 or more species, especially in the
    higher campos rupestres habitats embedded within the Cerrado
    s.l. (see e.g. Mimosa Koenen et al 2013 S Afr J Bot; Calliandra de
    Souza et al 2014 Taxon 62: 1201-1220; Chamaecrista Rando et al
    2016 Int J Plant Sci 177: 3-17).}

\item \textit{P4, column 1, lines 415-419. Much more significant
    perhaps than niche conservatism in this context could be simply
    the geographical adjacency of the HD, Himalaya and QTP, i.e. lack
    of strong dispersal limitation. Again, the contrast with the Andes
    where dispersal limitation and phylogenetic niche conservatism
    seem to have been important is striking.}

\item \textit{P3, column 1, lines 300-305. While the authors perhaps
    imply multiple factors in combination as drivers of
    diversification, the idea that it is indeed combinations of
    extrinsic and intrinsic factors (confluences and synnovations
    sensu Donoghue \& Sanderson, 2015. New Phytologist) which drive
    radiations perhaps merits greater clarity here. For Pedicularis it
    seems very likely that it is a combination of uplift,
    physiographic resource heterogeneity, geographical isolation and
    recurrent divergence of floral traits associated with pollinator
    sharing and reproductive interference are implicated.}

\item \textit{One important point to note is the central use of the
    BAMM method in one component of this study. Recently, BAMM has
    come in for sharp criticism by Moore et al. (2016) in this journal
    [Moore et al. 2016 Critically evaluating the theory and
    performance of Bayesian analysis of macroevolutionary
    mixtures. PNAS 113: 9569-9574]. However, it is not clear whether
    the potential flaws in BAMM highlighted by Moore et al. (2016) are
    really valid or not. One of the issues raised by Moore et
    al. (theoretical issues associated with the likelihood function)
    was addressed in version 2.5 of BAMM.  In the as yet unpublished
    response by Rabosky (http://bamm-project.org/replication.html),
    question marks about some of the Moore et al .(2016) analyses are
    raised.  Pending further insights into these questions, my view on
    this is that BAMM remains the best approach available for
    estimating diversification rates on all branches across a
    phylogeny, assuming version 2.5 was used. The manuscript does not
    state which version of BAMM was used. This needs to be clarified
    and if an earlier version was used, the analyses should be re-run
    using BAMM version 2.5.}

  We used BAMM version 2.5, the current version when the analyses were
  run (December 2015), and have clarified this in the text (XXX). That
  version corrected the only issue raised by Moore et al.\ that is
  relevant to this study: the apparent sensitivity of the inferred
  number of rate shifts to the prior. We now state the priors we used
  (XXX): for all clades, we set \texttt{expectedNumberOfShifts = 1};
  and for those larger than 500 total species (\textit{Allium},
  Delphineae, Polygoneae, Primulaceae, \textit{Rhododendron}, and
  Grossulariaceae; Table S1) we also tested
  \texttt{expectedNumberOfShifts = 10}. There were no substantial
  differences so we report the results for the more conservative
  prior, \texttt{expectedNumberOfShifts = 1}.

  The other issues raised in the Moore et al.\ paper do not provide
  any reason to think that our BAMM results are systematically biased
  with respect to our hypothesis. The main question would be whether
  they failed to reveal shifts that are in fact supported by the
  data---shifts that potentially are associated with geographic
  occupancy. We cannot see any reason to suspect this to be the case.

  Since the BAMM results are largely uninformative with respect to the
  goals of our study, and (as we argue in the text) the method does
  not appear not very well-suited to the data (due to the general
  scarcity of endemic radiations), we do not feel it is necessary to
  call attention to the Moore et al.\ controversy in the main text.

\item \textit{Methods - estimating diversification rates. It is stated
    that assignment of unsampled diversity for Rhododendron and Acer
    was based on the sectional classifications of those
    genera. However, no mention of how this was done for Allium (with
    47\% sampling), Rosa (58\%) of Thalictrum (69\%). Please clarify
    in more detail.}

  We have revised the Methods section (XXX) and added a section to the
  SI text (XXX) to explain our sampling for BAMM. We now cite the
  literature sources of our sampling fractions.

\item \textit{Abstract, last sentence. This may be the first
    quantitative evidence supporting a mountain uplift-driven
    diversification for the Hengduan mountains, but it is not the
    first more generally, with several recent studies addressing this
    issue for the Andes (e.g. Lagomarsino et al 2016. New Phytologist
    210: 1430-1442, and accompany commentary Hughes, 2016 New
    Phytologist 210: 1152-1154). Clarify.}

\item \textit{P1, second paragraph main text. I think citing some
    references to uplift-driven diversification would be in order in
    order to set this in its wider context, e.g. Hoorn et
    al. 2013. Nature Geosciences 6: 154; Lagomarsino et al 2016. New
    Phytologist 210: 1430-1442, and accompany commentary Hughes, 2016
    New Phytologist 210: 1152-1154; Merckx et al. 2015. Nature 524:
    347-350 etc.}

\item \textit{P1, line 51. Sentence not clear. Why tropical and
    Mediterranean? Need to state that most biodiversity hotspots are
    in these areas, and also worth stating here that the Hengduan
    mountain region is one of the richest temperate floras in the
    world. Clarify.}

\item \textit{Given the critical importance of the delimitation of the
    Hengduan Mountains as a distinct and separate area from the
    remainder of the QTP / Himalaya in this study (see above), an
    additional small figure with a map showing the location of these
    different areas would be extremely helpful for the reader. Yes,
    the map in FigS1 to some extent serves this purpose, but what I am
    envisaging is a map showing clearly the locations of the QTP /
    Himalaya and Hengduan.}

\item \textit{The area reconstructions for Pinaceae (Fig. S2K) look
    very odd, with large parts of the phylogeny coloured pale green =
    India where there are few Pinaceae, whilst there are very few
    terminals assigned to North America where a substantial proportion
    of Pinaceae diversity is found. Looks like there is a mistake
    here.}

  Thank you for catching this, it arose from an error in one of the
  scripts we wrote to automate running the analyses and making the
  figures. We have re-created all the reconstruction figures and the
  Pinaceae figure is corrected.

\item \textit{The diversification rate estimates and lack of any rate
    shifts for Primulaceae do not coincide with the previous analysis
    of this family by de Vos et al 2014. Proc Royal Soc B 281. Please
    check and see why this is and justify. In addition to the detailed
    notes on fossil constraints included in supporting information,
    notes about the results of the individual BAMM analyses should
    also be included to clarify issues like the Primulaceae example.}

  The BAMM estimates of net diversification rate depicted in Figure
  SXXX suggest that across most of the phylogeny, the rate is between
  about 0.15 and 0.2. This is not very different from the rates found
  by MEDUSA in de Vos et al.\ Figure 2, if you consider the 2 largest
  partitions where the values are 0.08 and 0.28, respectively.

  The reasons why BAMM does not find strong support for a shift in
  rates is a difficult question to answer because (1) our tree was
  different, as we sampled more species than de Vos et al. (354 vs.\
  265), and (2) they did not use BAMM, but instead analyzed
  diversification rates using MEDUSA, SymmeTREE, BiSSE, and
  BayesRate. Of those methods, only MEDUSA and SymmeTREE are
  comparable to BAMM in the sense of inferring the locations of rate
  shifts without \textit{a priori} hypotheses---in the case of de Vos
  et al., that faster diversification is associated with
  heterostyly. But the methods are otherwise very different in their
  model assumptions and algorithmic details. Unfortunately the
  phylogeny from de Vos et al.\ is not available on TreeBASE (despite
  what is stated in the paper) so we are unable to analyze it with
  BAMM for comparison.

\item \textit{None of the phylogenetic analyses include any indication
    of levels of support. It is certain that in many cases there is
    weak or no support for many elements of topologies presented,
    raising issues about over-interpretation of the biogeographical
    relationships of species. Support values should be indicated on
    the phylogenies and some discussion of this issue included under
    methods.}

  We do not indicate clade support in any tree figures because they
  are already visually cluttered, showing reconstructed ancestral
  ranges and diversification rates. In the text, it did not seem
  necessary to discuss clade support because our analyses were
  designed explicitly to account for phylogenetic uncertainty by
  making biogeographic inferences from replicated draws from the
  posterior distributions of trees for each clade. This is explained
  in Methods sections ``Ancestral range reconstruction'' and
  ``Regional assembly processes through time'', which we have edited
  to make this clearer. Our results thus do not condition on any
  particular topologies or branch lengths, and reflect the levels of
  clade support (and confidence in divergence times) provided by the
  sequence data.

\item \textit{Table S1. It looks as though the columns for total and
    sampled species for the temperate/boreal east Asian taxa have been
    reversed. Please check and rectify.}

  We have fixed this error.

\item \textit{Significance Statement, last sentence better: prevalence
    of mountains as global biodiversity hotspots.}

  We made this change.

\item \textit{P1, line 77, omit 'quite'}

  We made this change.

\item \textit{P1, line 96 better: .... in situ diversification has
    been clearly favored......}

  We made this change.

\item \textit{21. FigS2F appears twice}

  We have fixed this error.

\end{enumerate}

% \textbf{Reviewer \#2}

\textbf{Reviewer \#3}

\begin{enumerate}

\item \textit{It will be important to perform analyses of the 16
    lineages of angiosperm data alone. I wonder the impact of the
    older stem lineages of the Pinaceae (conifer lineage) in the
    analyses. Also Pinaceae and the fern lineage may have very
    different colonization strategies. Removing them from the analyses
    as an alternative seems important for the comparative analyses.}

  We redid the biogeographic analyses with only angiosperms (the 16
  original clades plus \textit{Lilium}, see below) and present them in
  Figures SXXX. The results do not change. The only appreciable
  difference is that the time scale of biogeographic events is
  shortened, as the earliest biogegraphic events, from about 180--100
  Ma, are all in Pinaceae.

\item \textit{The authors treated the Qinghai-Tibetan Plateau (QTP)
    and the Himalayas as one biogeographic area of endemism. I would
    like to see results of treating them separately.}

  Despite the uncertainty and scarcity of information needed to
  distinguish the Himalayas and QTP proper in terms of plant species
  distributions, we recoded our data set, treating them as separate
  areas. We scored 210 out of 4,668 species (4.5\%) as occurring in
  the QTP. We then generated new ancestral range reconstructions and
  estimated rates of dispersal and \textit{in situ} diversification
  through time. The results are summarized in Figures SXXX. They show
  that

\item \textit{The QTP-Himalayas-Hengduan Mountains have received so
    much attention in the last few years. Some lineages seem to have
    not been included, but they would fit perfectly the selection
    criteria, e.g., Lilium (Liliaceae), Juniperus (Cupressaceae),
    Caragana and close relatives (Fabaceae), to list just a few. I
    encourage the authors to include more of the lineages to more
    fully test the hypothesis of the diversification history in this
    fascinating region.}

\item \textit{Also, issues concerning phylogenetic uncertainty should
    be addressed. The results may not be as certain as it's
    presented. If so, the readers should know.}

\end{enumerate}

\textbf{Other changes:}

\begin{enumerate}

\item In Table S1, corrected global diversity of \textit{Rhodiola} to
  70 species based on Zhang et al.\ (2014) and re-ran BAMM with the
  updated global sampling fraction (0.81). This did not affect the
  results.

\end{enumerate}


Thank you for your consideration.

%\signature{Richard Ree}

\closing{Sincerely,\\
  \bigskip
  Richard Ree\\
  +1 312-665-7857\\
  rree@fieldmuseum.org\\
  \medskip on behalf of co-author Yaowu Xing
  }

% \encl{[ENCLOSURE LISTING]}

\end{letter}
\end{document}


%%% Local Variables: 
%%% mode: tex-pdf
%%% TeX-master: t
%%% End: 
