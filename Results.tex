\section{Results}

\subsection{Contrasting histories of floristic assembly}

Reconstructed biogeographic histories of the selected clades reveal distinct patterns of floristic assembly across regions when plotted as the cumulative number of colonization and \textit{in situ} speciation events through time (Fig.~\ref{fig:cumevents}). Given geological and paleontological evidence that the Hengduan Mountains are younger than the Himalayas-QTP, we expected to find that its flora was assembled more recently. Instead, we found surprisingly deep phylogenetic histories of Hengduan occupancy, with initial assembly of its flora pre-dating that of the Himalayas-QTP. In more than 95\% of the pseudoreplicated joint histories, the earliest colonization event occurs by 88 Ma for the Hengduan Mountains and by 56 Maa for the Himalayas; \textit{in situ} speciation begins later, starting by 35 Ma and 20 Ma, respectively. By contrast, in East Asia, which was commonly reconstructed as the root ancestral area for clades, \textit{in situ} speciation initially occurs by 106 Ma and colonization initially occurs by 78 Ma.

For all regions, species accumulation by each process is approximately exponential (log-linear) following initiation until about the last 8 Ma. During this earlier "constant" phase, assembly in both the Hengduan Mountains and Himalayas-QTP is dominated by colonization, but the rate of increase in \textit{in situ} speciation is faster than for colonization. By contrast, in temperate/boreal East Asia the dominant process is \textit{in situ} speciation, but the rates of increase in both processes are roughly equal. Following the constant phase, the assembly dynamics of the Hengduan Mountains and Himalayas diverge considerably, with relatively little change in temperate/boreal East Asia. In the Hengduan Mountains, the cumulative number of \textit{in situ} speciation events overtakes that of colonization around 8 Ma, and by the present, the median number is 496 versus 345. In the Himalayas-QTP, \textit{in situ} speciation never overtakes colonization, and by the present the median number of events is 192 versus 416. The relative contribution of \textit{in situ} speciation to floristic assembly is thus $496/(496+345) = 0.590$ for the Hengduan Mountains, and $192/(192+416) = 0.316$ for the Himalayas-QTP. In other words, \textit{in situ} speciation has contributed about twice as much to the assembly of the Hengduan Mountains flora as it has to the Himalayas-QTP flora, especially since the late Miocene.

\subsection{Regional rates of \textit{in situ} speciation and dispersal through time}

The rate of \textit{in situ} speciation for the Hengduan Mountains region, in events per resident lineage per Ma, increased almost twofold over the past 10 Ma, while for the Himalayas-QTP, it remained more or less constant (Fig.~\ref{fig:speciation}). Dispersal rates between the Hengduan Mountains and Himalayas-QTP increase gradually over the last 4--5 Ma, and sharply in the last 1 Ma. By contrast, dispersal from each region to temperate/boreal East Asia increases only slightly in the same time period (Fig.~\ref{fig:dispersal}).

\subsection{Macroevolutionary regime shifts are decoupled from dispersal}

BAMM analyses suggest that shifts in diversification regime are rarely coincident with geographic movements (Fig.~S2), as might be expected if dispersal events commonly established rapid endemic radiations. However, uncertainty in the precise phylogenetic locations of dispersal events and shifts in diversification regime make it difficult to associate bioeographic and macroevolutionary events.