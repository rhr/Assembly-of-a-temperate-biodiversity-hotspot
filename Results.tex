\section{Results}

\subsection{Contrasting histories of floristic assembly}

Reconstructed biogeographic histories of the selected clades reveal distinct patterns of floristic assembly across regions when plotted as the cumulative number of colonization and \textit{in situ} speciation events through time (Fig.~\ref{fig:cumevents}). Given geological and paleontological evidence that the Hengduan Mountains are younger than the Himalayas-QTP, we expected to find that its flora was assembled more recently. Instead, we found surprisingly deep phylogenetic histories of Hengduan occupancy, with initial assembly of its flora pre-dating that of the Himalayas-QTP. In more than 95\% of the pseudoreplicated joint histories, the earliest colonization event occurs by 88 Ma for the Hengduan Mountains and by 56 Maa for the Himalayas; \textit{in situ} speciation begins later, starting by 35 Ma and 20 Ma, respectively. By contrast, in East Asia, which was commonly reconstructed as the root ancestral area for clades, \textit{in situ} speciation initially occurs by 106 Ma and colonization initially occurs by 78 Ma.

For all regions, species accumulation by each process is approximately exponential (log-linear) following initiation until about the last 8 Ma. During this earlier "constant" phase, assembly in both the Hengduan Mountains and Himalayas-QTP is dominated by colonization, while in temperate/boreal East Asia the dominant process is \textit{in situ} speciation. Following this phase, the assembly dynamics of the Hengduan Mountains and Himalayas diverge considerably, with relatively little change in temperate/boreal East Asia. In the Hengduan Mountains, the cumulative number of \textit{in situ} speciation events overtakes that of colonization around 8 Ma, and by the present, the median number is 496 versus 345. In the Himalayas-QTP, \textit{in situ} speciation never overtakes colonization, and by the present the median number of events is 192 versus 416. The relative contribution of \textit{in situ} speciation to floristic assembly is thus $496/(496+345) = 0.590$ for the Hengduan Mountains, and $192/(192+416) = 0.316$ for the Himalayas-QTP. In other words, \textit{in situ} speciation has contributed about twice as much to the assembly of the Hengduan Mountains flora as it has to the Himalayas-QTP flora, especially since the late Miocene.

\subsection{Regional rates of \textit{in situ} speciation and dispersal through time}

The rate of \textit{in situ} speciation for the Hengduan Mountains region, in events per resident lineage per Ma, increased almost twofold over the past 10 Ma, while for the Himalayas-QTP, it remained more or less constant (Fig.~\ref{fig:speciation}). Dispersal rates between the Hengduan Mountains and Himalayas-QTP increase gradually over the last 4--5 Ma, and sharply in the last 1 Ma. By contrast, dispersal from each region to temperate/boreal East Asia increases only slightly in the same time period (Fig.~\ref{fig:dispersal}).

\subsection{Shifts in diversification rate}

Seven out of the 18 clades showed strong evidence (Bayes factor > 15) for one or more shifts in diversification regime (Table~\ref{table:bammbayesfactors}). However, inspection of branch-specific evidence (the cumulative probability of occurring in the credible set of shift configurations, and the marginal odds ratio in favor of a shift) in light of ancestral-range reconstructions does not reveal any clear geographic patterns in the phylogenetic locations of regime shifts (Fig.~S2). That is, across clades, macroevolutionary jumps in diversification rate are not obviously associated with unambiguous colonization events or occupancy of any particular region. A notable exception to this general observation is \emph{Rhododendron}, in which reconstructions from BAMM and Lagrange suggest that  in a window of about 9--15 Ma, net diversification increased independently in 2 clades that each originated in the Hengduan region and are currently dominated by Hengduan species (Fig.~\ref{fig:rhododendron}).


%%% Local Variables:
%%% mode: latex
%%% TeX-master: "master"
%%% End:
