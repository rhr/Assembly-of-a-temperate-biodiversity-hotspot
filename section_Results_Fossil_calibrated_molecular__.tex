\section{Results}

Fossil-calibrated molecular chronogram distributions were generated for 16 clades of angiosperms and one each of gymnosperms (Pinaceae) and ferns (Microsorioideae). The combined taxon sample included a total of 3,266 ingroup species, of which 899 occur in the Hengduan Mountains, 628 in the Himalayas, and 1,165 in temperate East Asia, representing XX\%, YY\%, and ZZ\% of the vascular flora of each region (SI REF).

Reconstructed biogeographic histories of the selected clades reveal distinct patterns of floristic assembly across regions when plotted as the cumulative number of \textit{in situ} speciation and colonization events through time (FIG: cumulative events). Given geological and paleontological inferences that the Hengduan Mountains are younger than the Himalayas, we expected to find that its flora was assembled more recently. Instead, we found surprisingly deep phylogenetic histories of Hengduan occupancy, with initial assembly of the Hengduan Mountains flora pre-dating that of the Himalayas. In $>$95\% of the pseudoreplicated joint histories, the earliest colonization event occurs by 88 Mya for the Hengduan Mountains and by 56 Mya for the Himalayas; \textit{in situ} speciation begins later, starting by 35 Mya and 20 Mya, respectively. By contrast, in East Asia, which was commonly reconstructed as the root ancestral area for clades, \textit{in situ} speciation initially occurs by 106 Mya and colonization initially occurs by 78 Mya.

For all regions, species accumulation by each process is seemingly exponential (log-linear) following initiation until about the last 8 My. During this earlier "constant" phase, assembly in both the Hengduan Mountains and Himalayas is dominated by colonization, but the exponential rate of accumulation by \textit{in situ} speciation is faster than for colonization. By contrast, in East Asia the dominant process is \textit{in situ} speciation.  After 10 Mya, the assembly dynamics of each region become very distinct. In East Asia there is relatively little change in assembly dynamics. In the Hengduan Mountains, the cumulative number of \textit{in situ} speciation events overtakes that of colonization around 8 Mya, and by the present, the median number is 496 versus 345. In the Himalayas, \textit{in situ} speciation never overtakes colonization, and by the present the median number of events is 192 versus 416. The relative contribution of \textit{in situ} speciation to floristic assembly is thus $496/(496+345) = 0.590$ for the Hengduan Mountains, and $192/(192+416) = 0.316$ for the Himalayas. In other words, \textit{in situ} speciation has contributed approximately twice as much to the assembly of the Hengduan flora as it has to the Himalayan flora.

\subsection{Rates of \textit{in situ} speciation through time}

To understand in more detail the contrasts in assembly dynamics between the Hengduan Mountains and Himalayas, we calculated an approximate rolling estimate of speciation rate through time as $\lambda(t) = s(t)/n(t-1)$, where $s(t)$ is the number of \textit{in situ} speciation events inferred in a 1-Myr time interval $t$ and $n(t-1)$ is the number of inferred lineages in the region in the previous time interval. Unlike the cumulative event curves in Figure (FIG:cumulative events), which do not account for the number of lineages available to each process at a given time, this shows that the rate of \textit{in situ} speciation for the Hengduan Mountains increased almost twofold over the past 10 Myr, while for the Himalayas, it has remained more or less constant (FIG:speciation).