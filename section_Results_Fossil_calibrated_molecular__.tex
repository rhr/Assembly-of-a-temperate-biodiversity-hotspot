\section{Results}

Fossil-calibrated molecular chronogram distributions were generated for 16 clades of angiosperms and one each of gymnosperms (Pinaceae) and ferns (Microsorioideae). The combined taxon sample included a total of 3,266 ingroup species, of which 899 occur in the Hengduan Mountains, 628 in the Himalayas, and 1,165 in temperate East Asia. Across clades, the proportion of taxa sampled ranged from XX--YY\% (REF:SI Table).%, representing approximately XX\%, YY\%, and ZZ\% of the vascular flora of each region (SI REF).

Reconstructed biogeographic histories of the selected clades reveal distinct patterns of floristic assembly across regions when plotted as the cumulative number of \textit{in situ} speciation and colonization events through time (FIG: cumulative events). Given geological and paleontological inferences that the Hengduan Mountains are younger than the Himalayas, we expected to find that its flora was assembled more recently. Instead, we found surprisingly deep phylogenetic histories of Hengduan occupancy, with initial assembly of the Hengduan Mountains flora pre-dating that of the Himalayas. In more than 95\% of the pseudoreplicated joint histories, the earliest colonization event occurs by 88 Mya for the Hengduan Mountains and by 56 Mya for the Himalayas; \textit{in situ} speciation begins later, starting by 35 Mya and 20 Mya, respectively. By contrast, in East Asia, which was commonly reconstructed as the root ancestral area for clades, \textit{in situ} speciation initially occurs by 106 Mya and colonization initially occurs by 78 Mya.

For all regions, species accumulation by each process is seemingly exponential (log-linear) following initiation until about the last 8 Myr. During this earlier "constant" phase, assembly in both the Hengduan Mountains and Himalayas is dominated by colonization, but the rate of increase in \textit{in situ} speciation is faster than for colonization. By contrast, in East Asia the dominant process is \textit{in situ} speciation, but the rates of increase in both processes are roughly equal. Following the constant phase, the assembly dynamics of the Hengduan Mountains and Himalayas diverge considerably, with relatively little change in East Asia. In the Hengduan Mountains, the cumulative number of \textit{in situ} speciation events overtakes that of colonization around 8 Mya, and by the present, the median number is 496 versus 345. In the Himalayas, \textit{in situ} speciation never overtakes colonization, and by the present the median number of events is 192 versus 416. The relative contribution of \textit{in situ} speciation to floristic assembly is thus $496/(496+345) = 0.590$ for the Hengduan Mountains, and $192/(192+416) = 0.316$ for the Himalayas. In other words, the relative contribution of \textit{in situ} speciation to the assembly of the Hengduan flora is about twice that of the Himalayan flora.

\subsection{Rates of speciation and dispersal through time}

Accounting for change in the number of lineages available for speciation and dispersal in each region reveals additional contrasts between the Hengduan Mountains and Himalayas. The rate of \textit{in situ} speciation for the Hengduan Mountains increased almost twofold over the past 10 Myr, while for the Himalayas, it remained more or less constant (FIG:speciation). Dispersal rates between the Hengduan Mountains and Himalayas increase gradually over the last 4--5 Myr, and sharply in the last 1 Myr, compared to dispersal from either region to East Asia, which increases only slightly in the same time period (Figure FIG:dispersal). The sharp increase in dispersal rates between the Hengduan Mountains and Himalayas in the last million years is likely driven by the large number of range expansion events needed to explain species that are widespread across (i.e., shared by) these regions.