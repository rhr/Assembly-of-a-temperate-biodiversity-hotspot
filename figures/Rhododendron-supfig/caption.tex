\label{fig:rhododendron}
Reconstructions of ancestral geographic range (left) and net diversification rate (right) on the maximum clade credibility tree, with branch lengths set to posterior means, for \textit{Rhododendron}. Ancestral ranges are maximum-likelihood estimates at the start and end of each branch. Net diversification values are branch-segment means of the posterior distribution estimated by BAMM. Filled circles on the right indicate branches that appear in the 95\% credible set of distinct shift configurations, with the size and label of a circle indicating the cumulative probability of the branch over all configurations in the credible set. On the left, the marginal odds ratio for a shift in diversification regime along a branch is drawn for branches where the ratio exceeds 20. Geographic regions are coded as follows: HEN, Hengduan Mountains; HIM, Himalayas-QTP; EAS, temperate-boreal East Asia; SEA, Southeast Asia; CWA, central/western Asia; EUR, Europe; IND, India; AFR, Africa; NAM, North America; SAM, South America; AUS, Australasia. Hengduan species cluster primarily in 2 clades, both of which show evidence of ancestral shifts to higher diversification rate.