\subsection{Index of supporting information}

\begin{itemize}
\item Table S1: global and regional taxon sampling for each clade
\begin{table} 
    \begin{tabular}{ clade sampled_richness Total_richness sampled(Hengduan) total(Hengduan) sampled(Himalayas-QTP) total(Himalayas-QTP) sampled(East_Asia) total(East_Asia) c c }
         Acer  &  &  &  &  &  &  &  &  &  \\ 
         &  &  &  &  &  &  &  &  &  &  \\ 
         &  &  &  &  &  &  &  &  &  &  \\ 
         &  &  &  &  &  &  &  &  &  &  \\ 
         &  &  &  &  &  &  &  &  &  &  \\ 
         &  &  &  &  &  &  &  &  &  &  \\ 
         &  &  &  &  &  &  &  &  &  &  \\ 
         &  &  &  &  &  &  &  &  &  &  \\ 
         &  &  &  &  &  &  &  &  &  &  \\ 
         &  &  &  &  &  &  &  &  &  &  \\ 
         &  &  &  &  &  &  &  &  &  &  \\ 
         &  &  &  &  &  &  &  &  &  &  \\ 
         &  &  &  &  &  &  &  &  &  &  \\ 
         &  &  &  &  &  &  &  &  &  &  \\ 
         &  &  &  &  &  &  &  &  &  &  \\ 
         &  &  &  &  &  &  &  &  &  &  \\ 
         &  &  &  &  &  &  &  &  &  &  \\ 
         &  &  &  &  &  &  &  &  &  &  \\ 
    \end{tabular} 
\end{table}

\item Table S2: Genbank accession numbers
\item Materials and methods
\item Dataset S1: species ranges scored as presence-absence in 11 biogeographic regions (spreadsheet)
\item DO WE NEED THIS? Figure S1: 18 subfigures showing reconstructed ancestral states and diversification rates on mirror-image MCC trees
\end{itemize}