\section{Results}

The time-calibrated molecular phylogenies estimated for the 18 selected clades sampled a total of 3,266 ingroup species, of which 899 occur in the Hengduan Mountains, 628 in the Himalayas, and 1,165 in temperate East Asia. The proportion of species sampled was roughly the same across regions, in the range of 50--60\% (SI REF).

Reconstructed biogeographic histories of the selected clades revealed distinct patterns of floristic assembly across regions when plotted as the cumulative number of \textit{in situ} speciation and colonization events through time (FIG: cumulative events). Given geological and paleontological inferences that the Hengduan Mountains are younger than the Himalayas, we expected to find that its flora was assembled more recently. Instead, we found surprisingly deep phylogenetic histories of Hengduan occupancy, with initial assembly of the Hengduan Mountains flora pre-dating that of the Himalayas. In \>95\% of the pseudoreplicates, the earliest colonization event occurred by 88 Mya for the Hengduan Mountains and by 56 Mya for the Himalayas; \textit{in situ} speciation began later, starting by 35 Mya and 20 Mya, respectively. By contrast, in East Asia, commonly reconstructed as the root ancestral area for clades, \textit{in situ} speciation initially occurred by 106 Mya and colonization initially occurred by 78 Mya.

For both regions, accumulation of species by each process  increases exponentially. In the Hengduan Mountains, the cumulative contribution of \textit{in situ} speciation to regional diversity overtakes that of colonization around 15 Mya, after which both processes 

; by contrast, for temperate East Asia, it is dominated by \textit{in situ} speciation. 