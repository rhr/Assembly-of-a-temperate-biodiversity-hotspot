\section{Results}

The time-calibrated molecular phylogenies estimated for the 18 selected clades sampled a total of 3,266 ingroup species, of which 899 occur in the Hengduan Mountains, 628 in the Himalayas, and 1,165 in temperate East Asia. The proportion of species sampled was roughly the same across these regions, in the range of 50--60\% (SI REF).

The 3 focal regions exhibited distinct temporal patterns of floristic assembly by in situ speciation and colonization (FIG: cumulative events). Given the geological view that the Hengduan Mountains are younger than the Himalayas, we expected to find that its assembly was initiated more recently. Instead, our biogeographic reconstructions show surprisingly deep phylogenetic histories of Hengduan occupancy, with initial floristic assembly of the Hengduan Mountains pre-dating that of the Himalayas. In \>95\% of the pseudoreplicates, initial colonization of the Hengduan Mountains occurred by 88 Mya; for the Himalayas, it occurred by 56 Mya. For both regions, \textit{in situ} speciation is delayed, starting by 35 Mya and 20 Mya, respectively. For both regions, both processes occur at more or less constant rates, with the rate of speciation exceeding that of colonization. In the Hengduan Mountains, the cumulative contribution of \textit{in situ} speciation to regional diversity overtakes that of colonization around 15 Mya, after which both processes 

; by contrast, for temperate East Asia, it is dominated by \textit{in situ} speciation. 