\section{Discussion}

Our analysis is the first to make quantitative temporal inferences about the relative contributions of \textit{in situ} lineage diversification and colonization to the assembly of one of the world's richest temperate floras. Despite precedent in the literature for considering the Hengduan Mountains to be simply part of a greater biogeographic region that includes the Himalayas and QTP \citep[e.g.][]{Zhang2014,Nie2013,GaoY2013,Matuszak2016}, we instead find that the Hengduan region actually has a very distinct history of assembly that reflects its younger age.

We expected regional differences to reflect contrasting times of orogeny: in particular, the uplift-driven diversification hypothesis predicts that \textit{in situ} speciation increases with mountain-building activity. In this context the late Miocene (12--8 Ma) is an important reference point, as previous studies of geology and paleontology indicate that the Hengduan Mountains achieved their current height only after this time, while the Himalayas and central QTP did so before (REFS). Here, our phylogenetic inferences, which make no prior assumptions about the timing of geological events, show that after about 8 Ma, the rate of \textit{in situ} speciation increased in the Hengduan Mountains, yielding a remarkable inflection point at which cumulative speciation overtakes colonization. This indicates that the Hengduan Mountains flora has been assembled disproportionately by recent \textit{in situ} speciation that coincides temporally with independent estimates of orogeny, supporting (at least superficially) the uplift-driven diversification hypothesis.

We do not find a similar signature of accelerated \textit{in situ} speciation for the Himalayas-QTP region, as might be expected from rapid Himalayan orogeny during the early to middle Miocene (REFS: Searle 2011?)  % \citep{WangY2007,Mao2010}
. It is possible that no such pulse of diversification occurred, e.g., if uplift of the Himalayas was more gradual. Alternatively, we could simply lack the statistical power to detect it, due to the masking effects of lumping the older QTP with the younger Himalayas, the greater uncertainty associated with estimating clade ages and ancestral ranges in deeper time, and/or extinction and turnover in the Himalayas-QTP flora.

% From Fig.2, speciation rate of Himalayas also increased until around 8 Ma, and keep constant since then. This probably due to Himalayas already reached its current elevation.

The older age of the Himalayas-QTP predicts that its flora should have been an early source of lineage dispersal to the younger Hengduan Mountains, which acted as a colonization sink. However, we do not find any marked asymmetry in dispersal between these regions through the end of the Miocene; instead, rates in both directions are relatively low and increase only gradually until about 2--3 Ma, when both increase more sharply, but more so from the Hengduan Mountains to the Himalayas-QTP (Fig.~\ref{fig:dispersal}). That is, in the Quaternary, the Hengduan Mountains appear to have acted more as a biogeographic \textit{source} than a sink for the Himalayas-QTP. This may reflect phylogeographic evidence that the Hengduan Mountains flora was likely buffered from extinction during Quaternary glacial cycles, with subsequent westward range expansion/recolonization of the Himalayas and QTP \citep[e.g.,][]{WangBS2011,CunY2010}. 

%Surpringly, our results indicate Himalayas-QTP flora was assembled later (Fig **). Ideally, bring paleontological evidence to bear on assembly/diversification questions - but the record is sparse. The exact time of onset of current biome remains unclear.

\textbf{Lack of correspondence between dispersal and shifts in diversification.}---Given the result that \textit{in situ} speciation is the dominant process in the Hengduan Mountains since the late Miocene, the lack of evidence for macroevolutionary jumps in diversification rate associated with colonization of the region might seem surprising. However, across our phylogenies, dispersal is relatively frequent and clades tend not to be entirely endemic to a single region. This reflects the fact that our regions are not, in general, defined by differences in biome; despite their differences in age, the Hengduan Mountains, Himalayas, and QTP share broad physiographic similarities that presumably facilitate biogeographic exchange. Dispersal between regions should not necessarily require extensive ecophysiological adaptations that may be difficult to evolve \citep{Donoghue2014}, tempering expectations of evolutionary radiations within a region \citep[cf.][]{Hughes2006}. For this reason our data are perhaps not well-suited for BAMM, which models diversification shifts as relatively rare events. Moreover, the link between dispersal events and shifts in diversification regime is further obscured by the inherent uncertainty associated with inferring their phylogenetic locations.

The most appropriate interpretation may simply be that the signal of higher diversification in the Hengduan region is more or less diffuse across clades, and emerges only when their biogeographic histories are considered jointly. This helps explain why previous analyses of single clades have not yielded the pattern inferred here.

\textbf{Drivers of Hengduan diversification.}---How and why did \textit{in situ} diversification in the Hengduan Mountains increase since the late Miocene, and more specifically, to what extent was it driven by orogeny? Further studies are needed. One might look for contrasts between the Hengduan Mountains and adjacent regions in the evolution of ecological traits and environmental tolerances \citep[e.g.,][]{liu2016} across multiple clades. These may reveal, for example, the extent to which speciation within a region is associated with adaptive divergence and niche-filling \citep{price2014}, as opposed to nonadaptive processes such as genetic isolation and drift arising from topographic effects (vicariance, emergence of sky islands, etc.). These analyses would require denser and more fine-grained taxonomic and geographic sampling than was possible here, in addition to the requisite trait data.

% Aridification and global cooling as a result of orogeny (see reviews in \citealt{MiaoY2012}) may be an important factor influencing the diversification of plants on the QTP.

% Accounting for the effect of Pleistocene climate fluctuations

We suspect that evidence for common processes will prove elusive, as the proximate causes of diversification are likely to be idiosyncratic. For many clades, factors unrelated to uplift \emph{per se}, such as biotic interactions, are almost certain to have played important roles. For example, in \emph{Pedicularis} (Orobanchaceae), diversification of the many Hengduan species may have been facilitated by recurrent divergence in floral traits associated with pollinator sharing and reproductive interference \citep[e.g.,][]{eaton2012}. (REFS:Huang)

A coarser-grained view is that diversification in the Hengduan region reflects a macroevolutionary response to the rapid expansion of moist temperate montane and alpine habitats that resulted from the confluence of orogeny, cooling climates, and intensification of the summer monsoon in the late Miocene.

During the early to middle Miocene, the southern QTP was covered by coniferous and deciduous-leaved forests \citep{SunJ2014,LiH1976}, despite having already achieved its current elevation. Limited paleobotanical evidence shows that the Himalayas-QTP may have experienced a dramatic species/vegetation shift after the middle Miocene. even paleoelevational recontruction suggesting this regions already reached their current elevation by then \citep{Spicer2003}. The onset and diversification of alpine biome may be much later than the uplift.

The Hengduan Mountains were formed in the context of falling late-Miocene temperatures. The summer monsoon was already well-developed (and perhaps intensifying); the QTP and Central Asian interior were already arid. Renner: drier climates after 14 Ma - Antarctic glaciation (Clift 2006, Clift et al. 2015).



Distinguishing the effects of orogeny and climate. 

More generally, it seems reasonable to view uplift and climate change as primary drivers of expanding ecological opportunities for temperate-montane species in the Hengduan region in the late Miocene.

In the intro, posed the question as agnostic re: diversification over colonization. We find that diversification wins. The question then becomes: is this a general result, or specific to the unique circumstances of the Hengduan Mountains, and therefore unpredictable?

What do our results illustrate about the assembly of biodiversity in the Hengduan Mountains in particular, or biodiversity hotspots/mountains in general

General considerations: ecological opportunities. Age, area, and environmental factors (moisture, temperature, seasonality, soil characteristics) influencing energy-diversity expectations.

Age: Hengduan Mountains are young; within it, alpine biome even younger



 extant diversity, and historical diversification (to the extent it can be inferred), reflects regional carrying capacities through time. 

An important consideration in this context is climatic conditions over time---moisture and temperature in particular---that influence primary productivity,  

Density-dependent effects: Himalayas are older, are niches more filled relative to Hengduan Mountains?

Pleistocene effects: differential extinction. Young clades: Saussurea

% Perenniality, woodiness are something like ``classic'' traits associated with alpine radiation (Hughes and Atchison) - but our clades include many herbaceous species. Hybridization (Rhododendron: Milne). Sky islands (e.g., Solms-Laubachia): does population isolation lead to the proliferation of narrow endemics (that are more prone to extinction)?

% Younger clades like Saussurea - ecological opportunity of new alpine habitat?

% Predict that earliest clades are montane, most recent radiations are alpine. Not entirely the case. Pinaceae, Acer, Delphineae are montane, and colonize early. Alpine clades like Saussurea and Ligularia colonized later and with exeptional high rates (Saussurea has crown age c. 3 Myr, but has generated more than 400 species). But Meconopsis(true???), Saxifragaceae, and Polygoneae have many alpine members (Saxifragaceae and Polygoneae include many montane even low land species as well), and are the earliest clades to accumulate \textit{in situ} speciation events in the Hengduan Mountains.

% Notably, clades that occur primarily in the relatively arid alpine zone, such as \textit{Saussurea} and \textit{Rhodiola}, show faster diversification in the past 10 Ma than montane clades such as Primulaceae (s.s.) and \textit{Cyananthus}. However, precipitation is not a limiting factor in north-south-going Hengduan Mountains since moist air from the Indian Ocean can go through by the valleys, facilitating diversification of montane clades like \textit{Rhododendron} and Primulaceae (s.s.). 

Early uplift of the plateau: what was the biome of the proto-QTP during the Eocene, Oligocene---presumably forests? Lagrange reconstructions of Hengduan (and Himalayan) ancestry that pre-date the origins of these mountains may indicate high-elevation ancestors

how old is the alpine biome in HM, Him? Prediction: alpine flora: young species, recent colonists

Uplift-driven diversification is either a too-simplistic hypothesis that fails to acknowledge the numerous other factors that likely contribute to biological diversification in mountains, or a too-vague catchall phrase that encompasses them all.

% Here we have found temporal alignment of elevated diversification and orogeny, with evidence coming from the joint analysis of many clades, which suggests that uplift may have had a general effect.

\textbf{Comparisons to other hotspots.}---% \textit{In situ} diversification also implicated in the assembly of other biodiversity hotspots: the Andes, California Floristic Province (Kay and Lancaster?), Cape Floristic Region? Uplift-driven diversification: Andes, Japan?
How is the Hengduan Mountains region similar to other biodiversity hotspots? How does it differ? The Hengduan Mountains region is a young biodiversity hotspot. Young hotspots contradict the general expectation that diversity is a function of time (Fine and Ree).

Comparison to paramo study: net diversification only, no dispersal; how did they define ``paramo'' clades (and clades of other hotspots)?

Comparison with northern Andes: both are young mountain regions. May reveal general properties of uplift-driven diversification.

Rapid evolutionary radiations of plants have been documented in most major mountain ranges worldwide \citep[reviewed in][]{Hughes2015}, including the European Alps \citep{Roquet2013}, Andes \citep[e.g.][]{Hughes2006,Luebert2014}), Rocky Mountains \citep{DrummondC2012}, New Zealand \citep{Joly2014}, and the East African Rift mountains \citep{Linder2014}. Most of these have been dated to the last 5 Ma, conciding with or occurring much later than orogeny. Apparently, the Hengduan mountain radiation is older dating back to 8--10 Ma. Andes is comparable with the Himalayas-QTP and the Hengduan mountains in several ways. They habor the most alpine diversity among the mountain ranges. The initial uplift of Andes and QTP dated back to late Eocene \citep{Gregory-Wodzicki2000,Graham2009}, but plant radiations are more influenced by recent uplift (late Miocene and onwards) \citep{Hughes2013,Luebert2014,Hughes2015,Madrinan2013}. \citet{Hughes2013} pointed out that different biomes in the Neotropics may have different evolutionary histories. It is probably the case in the Himalayas-Hengduan Mountains as well. 

OTHER TAXA: BIRDS: parrotbills \citep{liu2016}

\textit{in situ} divergence and speciation along elevational gradients not important; assembly of Himalayan avifauna by colonization \citep{johansson2007}

HIGHLIGHT NEED TO STOP LUMPING WITH ``HIMALAYAS''

\textbf{Sources of error and bias.}---Taxon sampling: incomplete, and in some cases not geographically representative (Saxifragaceae). However, our focus is on relative dynamics: comparing \textit{in situ} speciation to colonization (not absolute rates of either process) and comparing the Hengduan Mountains to the Himalayas, and both to the rest of temperate East Asia. So we generally expect the results to be unbiased with respect to our hypotheses.

Granularity. To place the biogeographic assembly of the Hengduan Mountains and adjacent regions in a global context, it was necessary to adjust the granularity of the geographic coding of species. In particlar, we lumped the Himalayas with the rest of the Qinghai-Tibetan Plateau, creating a composite region of heterogeneous ages.


%%% Local Variables:
%%% mode: latex
%%% TeX-master: "master"
%%% End:
