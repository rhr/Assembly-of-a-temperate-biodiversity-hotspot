\section{Discussion}

Our analysis is the first to make quantitative temporal inferences about the relative contributions of \textit{in situ} lineage diversification and colonization to the assembly of one of the world's richest temperate floras. Despite precedent in the literature for considering the Hengduan Mountains to be part of a greater biogeographic region that includes the Himalayas and QTP (e.g. \citealt{Zhang2014,Nie2013,GaoY2013,Matuszak2016}), we find that the Hengduan region actually has a very distinct history of assembly.

We expected regional differences to reflect contrasting times of orogeny: in particular, the uplift-driven diversification hypothesis predicts that \textit{in situ} speciation increases with mountain-building activity. In this context the late Miocene (12--8 Ma) is an important reference point, as previous studies of geology and paleontology indicate that the Hengduan Mountains achieved their current height only after this time, while the Himalayas and central QTP did so before. Here, our neontological analyses show that after about 8 Ma, the rate of \textit{in situ} speciation increased in the Hengduan Mountains, resulting in a remarkable inflection point at which cumulative speciation overtakes colonization. This suggests that the Hengduan Mountains flora has been assembled disproportionately by recent \textit{in situ} speciation in a time interval that coincides temporally with rapid orogeny, supporting the uplift-driven diversification hypothesis. We do not find a similar signature of accelerated \textit{in situ} speciation for the Himalayas-QTP region that might correspond with an earlier period of rapid uplift \citep{WangY2007,Mao2010}. This suggests that orogeny in the Himalayas-QTP region was more gradual, and/or that our statistical inferences lack sufficient power of detection, due to the increasing uncertainty in clade age estimates and ancestral range reconstructions in deeper time.

% From Fig.2, speciation rate of Himalayas also increased until around 8 Ma, and keep constant since then. This probably due to Himalayas already reached its current elevation.

%Increasing aridification as a result of orogeny (see reviews in \citealt{MiaoY2012}) may be an important factor influencing the diversification of plants on the QTP. Notably, clades that occur primarily in the relatively arid alpine zone, such as \textit{Saussurea} and \textit{Rhodiola}, show faster diversification in the past 10 Ma than montane clades such as Primulaceae (s.s.) and \textit{Cyananthus}. However, precipitation is not a limiting factor in north-south-going Hengduan Mountains since moist air from the Indian Ocean can go through by the valleys which fascinate diversification of montane clades like \textit{Rhododendron} and Primulaceae (s.s.). 

The older age of the Himalayas-QTP predicts that its flora was assembled first and acted as a source of lineage dispersal, while the younger Hengduan Mountains acted as a colonization sink.

%Surpringly, our results indicate Himalayas-QTP flora was assembled later (Fig **). Ideally, bring paleontological evidence to bear on assembly/diversification questions - but the record is sparse. The exact time of onset of current biome remains unclear. Limited paleontological evidence show that Himalayas-QTP may have experienced a dramatic species/vegetation shift after the middle Miocene. During the early to middle Miocene, southern QTP was still covered by conifer and deciduous-leaved forests \citep{SunJ2014,LiH1976} even paleoelevational recontruction suggesting this regions already reached their current elevation by then \cite{Spicer2003}. The onset and diversification of alpine biome may be much later than the uplift.   

We infer roughly equal rates of dispersal between regions until only the last 1--2 Ma, when dispersal from the Himalayas to the Hengduan Mountains increases more than the converse. That the rate of dispersal from the Hengduan Mountains to the Himalayas increases in the last 4--5 Ma suggests that the Hengduan Mountains began acting as a biogeographic source of lineages around the same time as its formation. 

%Several phylogeographical studies have suggested that the Hengduan mountains were less affected by the glacier and an east-to-west dispersal route were found in several clades during the Quaternary (e.g. \citealt{WangBS2011,CunY2010}). 

\textbf{Sources of error and bias.}---Taxon sampling: incomplete, and in some cases not geographically representative (Saxifragaceae). However, our focus is on relative dynamics: comparing \textit{in situ} speciation to colonization (not absolute rates of either process) and comparing the Hengduan Mountains to the Himalayas, and both to the rest of temperate East Asia. So we generally expect the results to be unbiased with respect to our hypotheses.

\textbf{Lack of biogeographic signal in macroevolutionary shifts.}---Our results could be construed as somewhat contradictory. On one hand, ancestral range reconstructions show increasing dominance of \textit{in situ} speciation over colonization in the Hengduan Mountains since the late Miocene, while on the other, BAMM analyses show little evidence for shifts in diversification rate being associated with colonization of the Hengduan region.  might be expected if dispersal events commonly established rapid endemic radiations. However, uncertainty in the precise phylogenetic locations of dispersal events and shifts in diversification regime make it difficult to associate bioeographic and macroevolutionary events.

Comparable analyses of other regions. \textit{In situ} diversification also implicated in the assembly of other biodiversity hotspots: the Andes, California Floristic Province (Kay and Lancaster?), Cape Floristic Region? Uplift-driven diversification: Andes, Japan?

Rapid evolutionary radiations of plants have been documented in most major mountain ranges worldwide \citep[reviewed in][]{HughesC2015}, including the European Alps \citep{Roquet2013}, Andes \citep[e.g.][]{Hughes2006,Luebert2014}), Rocky Mountains \citep{DrummondC2012}, New Zealand \citep{Joly2014}, and the East African Rift mountains \citep{Linder2014}. Most of these have been dated to the last 5 Ma, conciding with or occurring much later than orogeny. Apparently, the Hengduan mountain radiation is older dating back to 8--10 Ma. Andes is comparable with the Himalayas-QTP and the Hengduan mountains in several ways. They habor the most alpine diversity among the mountain ranges. The initial uplift of Andes and QTP dated back to late Eocene \citep{Gregory-Wodzicki2000,Graham2009}, but plant radiations are more influenced by recent uplift (late Miocene and onwards) \citep{Hughes2013,Luebert2014,HughesC2015,Madrinan2013}. \citet{HughesC2013} pointed out that different biomes in the Neotropics may have different evolutionary histories. It is probably the case in the Himalayas-Hengduan Mountains as well. 

Which clades colonized first? Predict that earliest clades are montane, most recent radiations are alpine. Not entirely the case. Pinaceae, Acer, Delphineae are montane, and colonize early. Alpine clades like Saussurea and Ligularia colonized later and with exeptional high rates (Saussurea has crown age c. 3 Myr, but generated more than 400 species). But Meconopsis(true???), Saxifragaceae, and Polygoneae have many alpine members(Saxifragaceae and Polygoneae include many montane even low land species as well), and are the earliest clades to accumulate \textit{in situ} speciation events in the Hengduan Mountains.

%%% Local Variables:
%%% mode: latex
%%% TeX-master: "master"
%%% End:
