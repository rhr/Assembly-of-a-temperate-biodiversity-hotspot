
% LaTeX Template for a Letter or Correspondence
% 
% To use:
%
% Copy into a new file, replace all
% [BRACKETED UPPER CASE TEXT]
% with your own, then run the latex command on it.
% Use dvips to print the .dvi output
\documentclass[12pt]{letter}
\usepackage[utf8]{inputenc}
\usepackage{times}
\usepackage[unicode=true,colorlinks=true,urlcolor=blue,citecolor=black,linkcolor=black]{hyperref}
\usepackage{geometry}
\geometry{verbose,tmargin= 1.5in,bmargin= 1.5in,
          lmargin= 1in,rmargin= 1in}

% If the letter doesn't fit on one page, but you
% want it do, remove the ``%'' on the next line:
% \setlength{\textheight}{8in}

\begin{document}
\raggedright{}
% If you want headings on subsequent pages,
% remove the ``%'' on the next line:
% \pagestyle{headings}

\begin{letter}{ \\

}
\address{Integrative Research Center\\
  The Field Museum\\
  1400 South Lake Shore Drive\\
  Chicago, IL  60605-2496\\
  USA}

\opening{To the Editors:}

We respectfully submit our manuscript, ``Uplift-driven diversification
in the Hengduan Mountains, a temperate biodiversity hotspot,'' for
publication in \textit{PNAS}. It is the first study to directly test
the pervasive hypothesis that lineage diversification is accelerated
during periods of mountain uplift. It does so by quantifying, across
18 representative clades, the tempo and mode of floristic assembly for
a major biodiversity hotspot, the Hengduan Mountains of south-central
China. The analysis reveals a clear signature of accelerated
\textit{in situ} diversification that temporally corresponds with the
region's relatively recent uplift.

The results are of general scientific interest because they
demonstrate the temporal and spatial dynamics of biodiversity
accumulation in relation to geological processes, and help explain why
mountains are so prevalent among the world's biodiversity
hotspots. Moreover, the potential for future citation seems high,
because uplift-driven diversification is a frequently-invoked
hypothesis, and this study is the first to support it with
quantitative evidence. Finally, as a novel application of historical
biogeographic inference methods to questions of regional biodiversity
assembly, this study may serve as a template for analyses of other
hotspots.

The following people would be appropriate reviewers:

\begin{enumerate}

\item Adrien Favre, University of Leipzig ---
  \href{mailto:adrien.favre@uni-leipzig.de}{adrien.favre@uni-leipzig.de}

\item Jun Wen, Smithsonian Institution -- \href{mailto:wenj@si.edu}{wenj@si.edu}

\item Toby Pennington, Royal Botanic Garden Edinburgh ---
  \href{mailto:t.pennington@rbge.ac.uk}{t.pennington@rbge.ac.uk}

\item Colin Hughes, University of Zürich ---
  \href{mailto:colin.hughes@systbot.uzh.ch}{colin.hughes@systbot.uzh.ch}

\item Isabel Sanmartín, Real Jardin Botanico, CSIC ---
  \href{mailto:isanmartin@rjb.csic.es}{isanmartin@rjb.csic.es}

\end{enumerate}

Thank you for your consideration.

%\signature{Richard Ree}

\closing{Sincerely,\\
  \bigskip
  Richard Ree\\
  +1 312-665-7857\\
  rree@fieldmuseum.org\\
  \medskip on behalf of co-author Yaowu Xing
  }

% \encl{[ENCLOSURE LISTING]}

\end{letter}
\end{document}


%%% Local Variables: 
%%% mode: tex-pdf
%%% TeX-master: t
%%% End: 
