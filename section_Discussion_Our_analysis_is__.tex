\section{Discussion}

Our analysis is the first to make quantitative inferences about the relative contributions of \textit{in situ} lineage diversification and colonization to the assembly of one of the world's richest temperate floras. Despite precedent in the literature for considering the Hengduan Mountains to be merely part of a larger biogeographic region that includes the Himalayas, we find that the 2 regions actually have very distinct histories of assembly. We expected these differences to reflect contrasting times of orogeny, with 10 Mya being an important reference point: studies of geology and paleontology indicate that the Hengduan Mountains achieved their current height only after this time, while the Himalayas did so before. Our phylogenetic analysis of extant species indicates that after 10 Ma, the rate of \textit{in situ} speciation rose in the Hengduan Mountains but stayed flat in the Himalayas, matching the prediction that mountain-building, either directly or indirectly through increased seasonal monsoon activity, accelerated lineage diversification.