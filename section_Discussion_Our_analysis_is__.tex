\section{Discussion}

Our analysis is the first to make quantitative inferences about the relative contributions of \textit{in situ} lineage diversification and colonization to the assembly of one of the world's richest temperate floras. Despite precedent in the literature for considering the Hengduan Mountains and Himalayas to be a single biogeographic region, we find that these regions actually have very distinct histories of assembly. We expected their differences to reflect contrasting times of orogeny: in particular, the uplift-driven diversification hypothesis predicts that \textit{in situ} speciation increases with mountain-building activity.

In this context the late Miocene (10-5 Mya) is an important reference point, as previous studies of geology and paleontology indicate that the Hengduan Mountains achieved their current height only after this time, while the Himalayas did so before. Our neontological analyses show that after about 8 Ma, the rate of \textit{in situ} speciation increased in the Hengduan Mountains, resulting in a remarkable inflection point at which cumulative \textit{in situ} speciation overtakes colonization. This suggests that the Hengduan Mountains flora has been assembled disproportionately by recent \textit{in situ} speciation that coincides temporally with rapid orogeny, supporting the uplift-driven diversification hypothesis.

We do not find a similar signature of accelerated \textit{in situ} speciation for the Himalayas corresponding with an earlier period of rapid uplift (REFS PREDICTING RAPID UPLIFT OF HIMALAYAS). This suggests that Himalayan orogeny was more gradual, and/or that our statistical inferences lack sufficient power, due to the increasing uncertainty in clade age estimates and ancestral state reconstructions in deeper time.

The older age of the Himalayas predicts that its flora was assembled first and acted as a source of lineage dispersal, while the younger Hengduan Mountains acted as a colonization sink. However, we infer roughly equal rates of dispersal between regions until only the last 1 Ma, when dispersal from the Himalayas to the Hengduan Mountains increases more than the converse. That the rate of dispersal from the Hengduan Mountains to the Himalayas increases in the last 4--5 Ma suggests that the Hengduan Mountains began acting as a biogeographic source of lineages around the same time as its formation.% Tradiational floristic studies suggested that many groups are originated in the Hengduan Mountains and then dispersed into other Northern temperate regions \cite{LiEtLi1993} \cite{Wu1988}. However, our results indicated most of groups are originated in East Aisa in our analysis. They diversified rapidly in the Hengduan Mountains and migrated into other regions. 

Which clades colonized first? Predict that earliest clades are montane, most recent radiations are alpine. Not entirely the case. Pinaceae, Acer, Delphineae are montane, and colonize early. But Meconopsis, Saxifragaceae, and Polygoneae have many alpine members, and are the earliest clades to accumulate \textit{in situ} speciation events in the Hengduan Mountains.

Sources of possible error and bias. Taxon sampling: incomplete, and in some cases not geographically representative (examples?). However, our focus is on relative dynamics: comparing \textit{in situ} speciation to colonization (not absolute rates of either process) and comparing the Hengduan Mountains to the Himalayas, and both to the rest of temperate East Asia. So we generally expect the results to be unbiased with respect to our hypotheses.

Ideally, bring paleontological evidence to bear on assembly/diversification questions - but the record is sparse.

Comparable analyses of other regions. \textit{In situ} diversification also implicated in the assembly of other biodiversity hotspots: the Andes, California Floristic Province (Kay and Lancaster?), Cape Floristic Region? Uplift-driven diversification: Andes, Japan?
