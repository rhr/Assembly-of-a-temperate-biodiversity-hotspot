\section{Discussion}

Our analysis is the first to make quantitative inferences about the relative contributions of \textit{in situ} lineage diversification and colonization to the assembly of one of the world's richest temperate floras. Despite precedent in the literature for considering the Hengduan Mountains to be merely part of a larger biogeographic region that includes the Himalayas, we find that these regions actually have very distinct histories of assembly. We expected their differences to reflect contrasting times of orogeny: in particular, the uplift-driven diversification hypothesis predicts that \textit{in situ} speciation will increase with mountain-building activity.

In this context the late Miocene (10-5 Mya) is an important reference point, as previous studies of geology and paleontology indicate that the Hengduan Mountains achieved their current height only after this time, while the Himalayas did so before. Our neontological analyses show that after about 8 Ma, the rate of \textit{in situ} speciation increased in the Hengduan Mountains, resulting in a remarkable inflection point at which cumulative \textit{in situ} speciation overtakes colonization. This suggests that the Hengduan Mountains flora has been assembled largely by recent \textit{in situ} speciation that coincides temporally with rapid orogeny, supporting the uplift-driven diversification hypothesis.

We do not find a similar signature of accelerated \textit{in situ} speciation for the Himalayas corresponding to an earlier period of rapid uplift (REFS PREDICTING RAPID UPLIFT OF HIMALAYAS). This suggests that Himalayan orogeny was more gradual, and/or that our statistical inferences lack sufficient power to detect such an event in deeper time. The difficulty of accurately reconstructing ancestral states is well known (REFS) and uncertainty accumulates with time before the present. In this regard, the somewhat unexpected result that colonization of the Hengduan Mountains region appears to pre-date its uplift since the late Miocene should be regarded with caution.

Effects of unknown extinction and turnover

\subsection{Role of colonization}
The older age of the Himalayas predicts that its flora was assembled first and acted as a source of lineages for colonization of the younger Hengduan Mountains. However we see roughly equal rates of dispersal between regions until only the last 1 Ma, when dispersal from the Himalayas to the Hengduan Mountains increases more than the converse. 



Ideally, bring paleontological evidence to bear on assembly/diversification questions - but the record is sparse.

Sources of possible bias. Taxon sampling: incomplete, and in some cases not geographically representative (examples?). Also missing some conspicuous clades (still unpublished): Pedicularis, Corydalis. However, our focus is on relative dynamics: comparing in situ speciation to colonization (not absolute rates of either process) and comparing the Hengduan Mountains to the Himalayas. So we generally expect the results to be unbiased with respect to our hypotheses.

Comparable analyses of other regions: Andes, California Floristic Province (Kay and Lancaster?), Cape Floristic Region?
