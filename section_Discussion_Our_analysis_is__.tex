\section{Discussion}

Our analysis is the first to make quantitative inferences about the relative contributions of \textit{in situ} lineage diversification and colonization to the assembly of one of the world's richest temperate floras. Despite precedent in the literature for considering the Hengduan Mountains to be merely part of a larger biogeographic region that includes the Himalayas, we find that the 2 regions actually have very distinct histories of assembly. We expected these differences to reflect contrasting times of orogeny, with 10 Mya being an important reference point: studies of geology and paleontology indicate that the Hengduan Mountains achieved their current height only after this time, while the Himalayas did so before. Our phylogenetic neontological analyses indicate that after 10 Ma, the rate of \textit{in situ} speciation rose in the Hengduan Mountains but stayed flat in the Himalayas, matching the prediction that mountain-building, either directly or indirectly (e.g., through increased monsoon activity), accelerated lineage diversification.

Uplift-driven diversification hypothesis also supported by contrast in total contribution of each process: for Hengduan, speciation >> colonization; converse for Himalayas.

The older age of the Himalayas predicts that its flora was assembled first and acted as a source of lineages for colonization of the younger Hengduan Mountains. However we see roughly equal rates of dispersal between regions until only the last 1 Ma, when dispersal from the Himalayas to the Hengduan Mountains increases more than the converse. 

Somewhat surprising result that assembly was initiated earlier in the Hengduan Mountains, indicating slightly higher average rates of speciation and colonization in the Himalayas. However, early history highly uncertain due to nature of data (ancestral state inferences from phylogenies of only extant species, uncertainty in molecular clocks, etc.). Lack of data on extinction is critical - possible that turnover has been high in the Himalayas

Ideally, bring paleontological evidence to bear on assembly/diversification questions - but the record is sparse.

Sources of possible bias. Taxon sampling: incomplete, and in some cases not geographically representative (examples?). Also missing some conspicuous clades (still unpublished): Pedicularis, Corydalis. However, our focus is on relative dynamics: comparing in situ speciation to colonization (not absolute rates of either process) and comparing the Hengduan Mountains to the Himalayas. So we generally expect the results to be unbiased with respect to our hypotheses.
