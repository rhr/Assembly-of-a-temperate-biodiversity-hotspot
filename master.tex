\documentclass[12pt]{article}
\usepackage{natbib}
\usepackage{times}
\usepackage[T1]{fontenc}
\usepackage[utf8]{inputenc}
\usepackage[pdftex]{graphicx}
\usepackage{caption}
\captionsetup[figure]{justification=raggedright,labelfont=bf}
\usepackage{fullpage} % 1" margins
\usepackage{setspace}
\setstretch{1.5}
%\usepackage{lineno}
%\usepackage{sectsty}
%\sectionfont{\nohang\centering\normalsize\sc}   % capitalize initial letters
%\subsectionfont{\nohang\centering\normalsize\rm\em}

% eat the colon so figures are labeled but have blank captions
% \makeatletter
% \renewcommand\fnum@figure[1]{\figurename~\thefigure\ignorespaces}
% \makeatother

%% Article
\begin{document}
\raggedright
\parindent 0.5in

\section{Introduction}

% Assembly of biodiversity hotspots - when, how, why? 2 basic processes: dispersal (immigration) and in situ diversification (speciation). Reconstructing these dynamics over time, in the context of geological and climatological events, is important to understanding the drivers/causes of major patterns in biogeography and macroecology (e.g., latitudinal diversity gradient).

Understanding when and how regional biotas were assembled is essential to understanding why the global distribution of biodiversity is geographically uneven. Among global biodiversity hotspots---regions of unusually species high richness and endemism---the mountains along the southern edge of the Qinghai-Tibetan Plateau (QTP) are both unusual and enigmatic: unusual because they constitute the richest hotspot that is neither tropical nor Mediterranean in climate, and enigmatic because, despite increasing interest from biogeographers, the timing, tempo, and mode of its biotic assembly remains poorly understood.

The region can be subdivided into the richer Hengduan Mountains to the east, with a flora that includes ca.\ 12,000 species in ca.\ 500,000 km$^2$ (Boufford; Li, 1993 #2; Wu, 1988 #1), and the Himalayas to the west, with ca.\ 10,000 species in ca.\ 750,000 km$^2$ (REF).

The flora of the Hengduan Mountains region in particular has been the focus of an increasing number of phylogenetic studies that make a particular claim: namely, that lineage diversification has been driven by mountain uplift as India continues to collide with Eurasia.

None of these studies explicitly quantified diversification rates between HMH and other regions which makes hard to conclude whether extraordinary plant diversity in HMH is due to rapid diversification. Moreover, estimates of the time spans of these radiations vary widely (between 20-1.56 Ma), and in most cases are based only on secondary molecular-clock calibrations. Therefore, reconstructing the diversification rates dynamics across different clades will provide new insights into diversification processes in the HMH.


It remains untested that how much immigration by pre-adapted species from adjacent regions had contributed to the exceptional plant diversity in the HMH. Much effort has been applied to reconstruct geographic patterns for clades in the QTP and adjacent regions (). Different hypotheses have been proposed in explaining current biogeographic pattern. Some studies indicate that most of the clades originated in the QTP and adjacent regions and started to specify in other Northern Hemisphere regions (e.g. Zhang et al., 2007, 2009; Xu et al., 2010; Zhang et al., 2014(Rhodiola)). Alternatively, some clades originated in other regions and radiated in the QTP (e.g. Liu et al., 2002; Tu et al., 2010). However, little is known when, from where and to which extent immigration has contributed to the spectacular diversity in the HMH in relative to in situ speciation. Were these biogeographic patterns associated with geological and environmental changes?

Though the uplift history of QTP and adjacent regions is extremely complex and detailed uplift processes are still controversial (see review in {Wang, 2014 #216} and Deng and Ding, 2016). Geological evidence showed that the central QTP reached it current elevation probably as early as 51-40 Ma, at least by 28 Ma {Xu, 2013 #217}{Wang, 2014 #216} followed by an inside-out growth pattern. The Himalayan region may have reached its current elevation at round 11-9 Ma. Recent evidence indicated that the uplift of the Hengduan Mt was mainly after 10 Ma {Wang, 2012 #218}{Mulch, 2006 #81}{Sun, 2011 #43}. Therefore, the nascent Hengduan region surely had close biogeographic connections to pre-existing alpine environments to the north and west, from which mountain-adapted clades could have dispersed. For example, in Cyananthus (Campanulaceae), phylogenetic analysis suggests that vicariance triggered by the uplift of the QTP accelerated the divergence of regional clades (sections), and that the Hengduan Mountains were colonized from the Himalayas {Zhou, 2013 #94}. More complex pattern has been found in the genus Anaphalis, (Asteraceae) that Anaphalis first radiated in the eastern Himalayas (Hengduan Mt) and migrated into eastern Asia, western Himalayas, as well as into North America, and SE Asia. This means Hengduan Mt mainly act as both species cradle and source of other floras. A general picture is still lacking and quantitative analyses are needed to address the contribution of immigration.

"Uplift-driven diversification" hypothesis: The expectation is that uplift has a general effect on clades - that it increases the potential for range subdivision/isolation, as well as adaptive divergence/radiation along environmental gradients associated with mountain systems. General prediction: rates of in situ diversification are higher during periods of uplift. However, also possible that uplift, in exposing/creating new habitat (e.g., alpine), increases the opportunity for colonization/immigration. A priori, we might expect the buildup/assembly of a mountain flora to be driven by both processes.

To investigate the claim of "uplift-driven diversification", need to measure rates of in situ diversification and immigration of lineages over time. Also need to know when uplift occurred.

The question can be framed as a comparison of the Hengduan Mountains with adjacent regions. Is there a general tendency for clades to diversify faster in the Hengduan Mountains? To what extent is the accumulation of species through time driven by \textit{in situ} diversification versus immigration (source-sink dynamics)?

The principal comparison is to the Himalayas. Conventional wisdom is that the Himalayas were uplifted earlier, and are hence older, than the Hengduan Mountains. In the Himalayas we might thus expect clades to have generally deeper histories of in situ diversification, and to have acted as a source of lineages for Hengduan Mountains colonization.

Methods of inferring biogeographic dynamics over evolutionary timescales. Fossil record - track the diversity and occurrence of species/clades in space directly through time. Alternatively, use time-calibrated phylogenies of extant species, and inferences of historical biogeography - where were lineages in the past, where did they diversify, and when/how did they move between regions. Both approaches: issue of sampling - incompleteness, bias. Phylogenies/anc. range reconstruction: issue of extinction.

In this study we study the historical assembly of the Hengduan Mountains biodiversity hotspot using biogeographic inferences from multiple clades of vascular plants. Our primary goal is to test the hypothesis that its species-rich flora is the result of higher rates of situ diversification than immigration. A secondary hypothesis is that lineage-range dynamics in the Hengduan Mountains over time are qualitatively distinct from those in adjacent regions - that is, we wish to study the empirical justification for distinguishing it from the Himalayas. Multiple clades: time-calibrated phylogenies, species ranges. Ancestral range reconstruction yields inferences about when and how lineages moved and proliferated, integrated over uncertainty in topologies, node ages, and precise sequences of events.
\section{Materials and Methods}

\subsection{Clade selection and phylogeny reconstruction}

Our criteria were that each clade (1) included a substantial number of species that occur in the Hengduan Mountains, as well as their closest known relatives in other biogeographic regions; (2) had sufficient molecular data available to infer a phylogeny that was broadly representative of the clade's taxonomic diversity and geographic range; and (3) had fossil data suitable for molecular clock calibration or secondary calibrations inferred from fossil dated phylogenies. We found 18 clades---16 of angiosperms and one each of gymnosperms (Pinaceae) and ferns (Microsorioideae)---that met these criteria. The combined taxon sample included a total of 3,262 ingroup species, of which 899 occur in the Hengduan Mountains region, 624 in the Himalayas-QTP, and 1,165 in the remainder of temperate/boreal East Asia (see \textit{Delimitation of biogeographic regions}). Of these, about 370 species are shared by the Hengduan Mountains and Himalayas-QTP regions. Across clades, the proportion of species sampled ranged from 20--97\% globally and 40--97\% for the Hengduan Mountains region (Table S1).

For each clade, we assembled molecular sequence alignments (Table S2; \textit{Appendix 1}) and fossil calibration data, and used relaxed molecular clock models implemented in BEAST 1.8 \citep{Drummond2012} or 2.0 \citep{Bouckaert2014} to generate a Bayesian posterior sample of time-calibrated phylogenies and the associated maximum-clade-credibility tree.  Detailed information is provided in \textit{SI Materials and Methods}.

%, representing approximately XX\%, YY\%, and ZZ\% of the vascular flora of each region (SI REF).

\subsection{Inference of range evolution and lineage diversification}

\subsubsection{Delimitation of biogeographic regions}

We defined 11 biogeographic regions (Fig.~S1), balancing considerations of model complexity, the need to accommodate the global distributions of the selected clades, and the granularity of our species range data. Of primary importance was distinguishing the Hengduan Mountains region from the geologically older parts of the QTP, especially the Himalayas. To that end, our definition of the Hengduan Mountains region follows \citet{Boufford2014}, and is bounded to the west by the Yarlung Tsangpo River in eastern Xizang (Tibet), to the northwest by the high plateau in Qinghai, to the north by the Tao River in southern Gansu, to the east by the Sichuan Basin, and to the south by subtropical forests and the Yunnan–Guizhou Plateau. As a primary point of comparison, we defined the ``Himalayas-QTP'' region as including the plateau itself, the Himalayas to the south, the Kunlun Mountains to the north, and the Qilian Mountains to the northeast. The other regions we defined are: temperate/boreal East Asia (the eastern boreal part of Russia and temperate regions of East Asia, including Japan and Taiwan); central/western Asia (including the Xinjiang Uyghur Autonomous Region to the south of Kunlun Range); southeast Asia (including tropical regions of China, Malesia, and Papuasia); Australasia (Australia, New Zealand, and the southwestern Pacific); India (south of the Himalyas, including Sri Lanka); Africa; Europe; North America; and South America (south of the Panama Canal). We scored the geographic range of each sampled species as its presence or absence in these regions based on floras, online databases, and published data sets (\textit{Dataset S1}; see also \textit{SI Materials and Methods}).

\subsubsection{Ancestral range reconstruction}

We set up a DEC model of range evolution in Lagrange \citep{Ree2005,Ree2008}, specifying a region-adjacency matrix that defined the valid set of spatially contiguous ranges. We explicitly avoided placing any temporal constraints on dispersal, so that biogeographic inferences were independent of prior beliefs about the ages of the Hengduan Mountains or Himalayas-QTP regions. For each clade, we used the model to infer a distribution of phylogenetic histories of biogeographic events. A single history for a clade was generated by sampling a chronogram from its Bayesian posterior distribution, estimating the maximum-likelihood set of ancestral ranges (geographic speciation scenarios) at internal nodes, and randomly interpolating parsimonious sequences of dispersal and local extinction events along branches having different ancestral and descendant ranges. This yielded a complete chronology of biogeographic events, i.e., where and when ancestral lineages moved and underwent speciation and local extinction. To account for phylogenetic uncertainty, one history was generated for each of 500 chronograms drawn randomly from the posterior for each clade. For further details, see \textit{SI Materials and Methods}.

\subsubsection{Regional assembly processes through time}

We focused our analysis on the dynamics of the Hengduan Mountains region in comparison to the Himalayas-QTP and temperate East Asia. Our objective was to estimate, for each region, rates of \textit{in situ} diversification and colonization and their cumulative contributions to biotic assembly through time. Taking the biogeographic histories of clades from the ancestral-range analysis, we generated 500 pseudoreplicated joint histories---sets of one history drawn randomly from each clade's pool without replacement. From each joint history, we extracted the chronology of \textit{in situ} speciation, colonization, and local extinction events affecting the 3 regions of interest and binned them into 1 Myr periods. This allowed region-specific estimates of assembly processes through time. We calculated rolling estimates of \textit{in situ} speciation rates as $\lambda(t) = s(t)/n(t-1)$, where $s(t)$ is the number of \textit{in situ} speciation events inferred in a region in a 1-Myr period $t$ and $n(t-1)$ is the number of inferred lineages in the region in the previous period (the cumulative sum of \textit{in situ} speciation and colonization minus local extinction). We also calculated rolling rates of dispersal for each region as $d_{ij}(t) = c_{ij}(t)/n_i(t-1)$, where $c_{ij}(t)$ is the number of inferred colonization events of area $j$ from area $i$. For all estimates, confidence intervals (95\% quantiles) were calculated from the pseudoreplicated joint histories. Further details are provided in the \textit{SI Materials and Methods}.

\subsubsection{Geography-independent rates of diversification}

To complement the regional-process analysis and to better understand heterogeneity in rates of lineage diversification independently of geography, we generated Bayesian inferences of diversification rate for each clade using BAMM \cite{Rabosky2014}, which uses Markov chain Monte Carlo procedures to jointly estimate the number, parameters, and locations of distinct macroevolutionary regimes (rates of lineage birth and death, possibly time-dependent) on a phylogenetic tree. BAMM is able to account statistically for incomplete taxon sampling, so we assigned sampling fractions at the finest level of taxonomic resolution possible: this was generally at the level of genus, but at the level of section for \textit{Rhododendron} and \textit{Acer}. In cases where genera were not confidently resolved as monophyletic, we assigned sampling fractions at the higher taxonomic levels (tribe and subfamily). We ran each BAMM MCMC for 100 million generations with a sampling frequency of 1000 and assessed convergence by visually inspecting plots of the likelihood trace and calculating the effective sample size after discarding the first 10\% of the run as burn-in. We identified the 95\% credible set of distinct shift configurations and the overall best set of rate shifts given the data using the BAMMtools package in R \cite{Rabosky2014}.
\section*{Results}

\subsection*{Contrasting histories of floristic assembly}

Reconstructed biogeographic histories of the selected clades reveal
distinct patterns of floristic assembly across regions when plotted as
the cumulative number of colonization and \textit{in situ} speciation
events through time (Fig.~\ref{fig:cumevents}). Given geological and
paleontological evidence that the Hengduan Mountains are younger than
the Himalayas-QTP, we expected to find that its flora was assembled
more recently. Instead, we found surprisingly deep phylogenetic
histories of Hengduan occupancy, with initial assembly of its flora
pre-dating that of the Himalayas-QTP. In more than 95\% of the
pseudoreplicated joint histories, the earliest colonization event
occurs by 64 Ma for the Hengduan Mountains and by 59 Ma for the
Himalayas-QTP; \textit{in situ} speciation begins later, starting by
34 Ma and 19 Ma, respectively. By contrast, in East Asia, which was
commonly reconstructed as the root ancestral area for clades,
\textit{in situ} speciation initially occurs by 155 Ma and
colonization initially occurs by 60 Ma.

For all regions, species accumulation by each process is approximately
exponential (log-linear) following initiation until about the last 8
Ma. During this earlier "constant" phase, assembly in both the
Hengduan Mountains and Himalayas-QTP is dominated by colonization,
while in temperate/boreal East Asia the dominant process is \textit{in
  situ} speciation. Following this phase, the assembly dynamics of the
Hengduan Mountains and Himalayas diverge considerably, with relatively
little change in temperate/boreal East Asia. In the Hengduan
Mountains, the cumulative number of \textit{in situ} speciation events
overtakes that of colonization around 8 Ma, and by the present, the
median number is 508 versus 335. In the Himalayas-QTP, \textit{in
  situ} speciation never overtakes colonization, and by the present
the median number of events is 192 versus 411. The relative
contribution of \textit{in situ} speciation to floristic assembly is
thus $508/(508+335) = 0.603$ for the Hengduan Mountains, and
$192/(192+411) = 0.318$ for the Himalayas-QTP. In other words,
\textit{in situ} speciation has contributed almost twice as much to
the assembly of the Hengduan Mountains flora as it has to the
Himalayas-QTP flora, especially since the late Miocene.

\subsection*{Regional rates of \textit{in situ} speciation and
  dispersal through time}

The rate of \textit{in situ} speciation for the Hengduan Mountains
region, in events per resident lineage per Ma, increased almost
twofold over the past 10 Ma, while for the Himalayas-QTP, it remained
more or less constant (Fig.~\ref{fig:speciation}). Dispersal rates
between the Hengduan Mountains and Himalayas-QTP increase gradually
over the last 4--5 Ma, and sharply in the last 1 Ma. By contrast,
dispersal from each region to temperate/boreal East Asia increases
only slightly in the same time period (Fig.~\ref{fig:dispersal}).

\subsection*{Shifts in diversification rate}

Seven out of the 19 clades showed strong evidence (Bayes factor > 15)
for one or more shifts in diversification regime
(Table~\ref{table:bammbayesfactors}). However, inspection of
branch-specific evidence (the cumulative probability of occurring in
the credible set of shift configurations, and the marginal odds ratio
in favor of a shift) in light of ancestral-range reconstructions does
not reveal any clear geographic patterns in the phylogenetic locations
of regime shifts (Fig.~S2). That is, across clades, macroevolutionary
jumps in diversification rate are not obviously associated with the
colonization or occupancy of any particular region.

A notable exception to this general pattern is \emph{Rhododendron}, in
which results from BAMM and Lagrange suggest that in a window of about
9--15 Ma, net diversification increased independently in 2 clades that
each originated in the Hengduan region and are currently dominated by
Hengduan species (Fig.~\ref{fig:rhododendron}). In \emph{Saussurea},
an increase in net diversification is inferred about 1.7 Ma along the
stem of a branch containing most of the clade's Hengduan species
(Fig.~S2O). Similarly, in Delphineae, 2 separate increases are
inferred in clades that are ancestrally Hengduan and together hold
most of the Hengduan species; however the times of these rate shifts
(about 28 Ma and 37 Ma, respectively) pre-date the uplift of the
Hengduan Mountains (Fig.~S2D). Finally, in \emph{Isodon} and the
\emph{Ligularia-Cremanthodium-Parasenecio} complex, branch-specific
measures show one increase in net diversification for each clade in
the context of Hengduan ancestry about 7 and 5 Ma, respectively, but
in both cases, models with regime shifts are not supported by Bayes
factors, and the Hengduan Mountains region is reconstructed as
ancestral across most of the phylogeny, rendering the geographic
context of the shift less informative.

%%% Local Variables:
%%% mode: latex
%%% TeX-master: "master"
%%% End:

%\section{Discussion}

Our analysis is the first to make quantitative temporal inferences about the relative contributions of \textit{in situ} lineage diversification and colonization to the assembly of one of the world's richest temperate floras. Despite precedent in the literature for considering the Hengduan Mountains to be part of a greater biogeographic region that includes the Himalayas and QTP (e.g. \citealt{Zhang2014,Nie2013,GaoY2013,Matuszak2016}), we find that the Hengduan region actually has a very distinct history of assembly.

We expected regional differences to reflect contrasting times of orogeny: in particular, the uplift-driven diversification hypothesis predicts that \textit{in situ} speciation increases with mountain-building activity. In this context the late Miocene (12--8 Ma) is an important reference point, as previous studies of geology and paleontology indicate that the Hengduan Mountains achieved their current height only after this time, while the Himalayas and central QTP did so before. Here, our neontological analyses show that after about 8 Ma, the rate of \textit{in situ} speciation increased in the Hengduan Mountains, resulting in a remarkable inflection point at which cumulative speciation overtakes colonization. This suggests that the Hengduan Mountains flora has been assembled disproportionately by recent \textit{in situ} speciation in a time interval that coincides temporally with rapid orogeny, supporting the uplift-driven diversification hypothesis. We do not find a similar signature of accelerated \textit{in situ} speciation for the Himalayas-QTP region that might correspond with an earlier period of rapid uplift \citep{WangY2007,Mao2010}. This suggests that orogeny in the Himalayas-QTP region was more gradual, and/or that our statistical inferences lack sufficient power of detection, due to the increasing uncertainty in clade age estimates and ancestral range reconstructions in deeper time.

% From Fig.2, speciation rate of Himalayas also increased until around 8 Ma, and keep constant since then. This probably due to Himalayas already reached its current elevation.

%Increasing aridification as a result of orogeny (see reviews in \citealt{MiaoY2012}) may be an important factor influencing the diversification of plants on the QTP. Notably, clades that occur primarily in the relatively arid alpine zone, such as \textit{Saussurea} and \textit{Rhodiola}, show faster diversification in the past 10 Ma than montane clades such as Primulaceae (s.s.) and \textit{Cyananthus}. However, precipitation is not a limiting factor in north-south-going Hengduan Mountains since moist air from the Indian Ocean can go through by the valleys which fascinate diversification of montane clades like \textit{Rhododendron} and Primulaceae (s.s.). 

The older age of the Himalayas-QTP predicts that its flora was assembled first and acted as a source of lineage dispersal, while the younger Hengduan Mountains acted as a colonization sink.

%Surpringly, our results indicate Himalayas-QTP flora was assembled later (Fig **). Ideally, bring paleontological evidence to bear on assembly/diversification questions - but the record is sparse. The exact time of onset of current biome remains unclear. Limited paleontological evidence show that Himalayas-QTP may have experienced a dramatic species/vegetation shift after the middle Miocene. During the early to middle Miocene, southern QTP was still covered by conifer and deciduous-leaved forests \citep{SunJ2014,LiH1976} even paleoelevational recontruction suggesting this regions already reached their current elevation by then \cite{Spicer2003}. The onset and diversification of alpine biome may be much later than the uplift.   

We infer roughly equal rates of dispersal between regions until only the last 1--2 Ma, when dispersal from the Himalayas to the Hengduan Mountains increases more than the converse. That the rate of dispersal from the Hengduan Mountains to the Himalayas increases in the last 4--5 Ma suggests that the Hengduan Mountains began acting as a biogeographic source of lineages around the same time as its formation. 

%Several phylogeographical studies have suggested that the Hengduan mountains were less affected by the glacier and an east-to-west dispersal route were found in several clades during the Quaternary (e.g. \citealt{WangBS2011,CunY2010}). 

\textbf{Sources of error and bias.}---Taxon sampling: incomplete, and in some cases not geographically representative (Saxifragaceae). However, our focus is on relative dynamics: comparing \textit{in situ} speciation to colonization (not absolute rates of either process) and comparing the Hengduan Mountains to the Himalayas, and both to the rest of temperate East Asia. So we generally expect the results to be unbiased with respect to our hypotheses.

\textbf{Lack of biogeographic signal in macroevolutionary shifts.}---Our results could be construed as somewhat contradictory. On one hand, ancestral range reconstructions show increasing dominance of \textit{in situ} speciation over colonization in the Hengduan Mountains since the late Miocene, while on the other, BAMM analyses show little evidence for shifts in diversification rate being associated with colonization of the Hengduan region.  might be expected if dispersal events commonly established rapid endemic radiations. However, uncertainty in the precise phylogenetic locations of dispersal events and shifts in diversification regime make it difficult to associate bioeographic and macroevolutionary events.

Comparable analyses of other regions. \textit{In situ} diversification also implicated in the assembly of other biodiversity hotspots: the Andes, California Floristic Province (Kay and Lancaster?), Cape Floristic Region? Uplift-driven diversification: Andes, Japan?

Rapid evolutionary radiations of plants have been documented in most major mountain ranges worldwide \citep[reviewed in][]{HughesC2015}, including the European Alps \citep{Roquet2013}, Andes \citep[e.g.][]{Hughes2006,Luebert2014}), Rocky Mountains \citep{DrummondC2012}, New Zealand \citep{Joly2014}, and the East African Rift mountains \citep{Linder2014}. Most of these have been dated to the last 5 Ma, conciding with or occurring much later than orogeny. Apparently, the Hengduan mountain radiation is older dating back to 8--10 Ma. Andes is comparable with the Himalayas-QTP and the Hengduan mountains in several ways. They habor the most alpine diversity among the mountain ranges. The initial uplift of Andes and QTP dated back to late Eocene \citep{Gregory-Wodzicki2000,Graham2009}, but plant radiations are more influenced by recent uplift (late Miocene and onwards) \citep{Hughes2013,Luebert2014,HughesC2015,Madrinan2013}. \citet{HughesC2013} pointed out that different biomes in the Neotropics may have different evolutionary histories. It is probably the case in the Himalayas-Hengduan Mountains as well. 

Which clades colonized first? Predict that earliest clades are montane, most recent radiations are alpine. Not entirely the case. Pinaceae, Acer, Delphineae are montane, and colonize early. Alpine clades like Saussurea and Ligularia colonized later and with exeptional high rates (Saussurea has crown age c. 3 Myr, but generated more than 400 species). But Meconopsis(true???), Saxifragaceae, and Polygoneae have many alpine members(Saxifragaceae and Polygoneae include many montane even low land species as well), and are the earliest clades to accumulate \textit{in situ} speciation events in the Hengduan Mountains.

%%% Local Variables:
%%% mode: latex
%%% TeX-master: "master"
%%% End:


\bibliography{bibliography/biblio}
\bibliographystyle{ecol_let}

\begin{figure}
\begin{center}
\includegraphics[width=.99\textwidth]{figures/figure_cumulative_events/figure_cumulative_events.pdf}
\end{center}
\caption{Assembly of regional floras by colonization and \textit{in situ} speciation events in 18 plant clades, inferred from ancestral-range reconstructions on time-calibrated molecular phylogenies. Shaded regions indicate the 5--95\% quantile intervals for the cumulative number of events through time from 500 pseudoreplicated joint biogeographic histories designed to account for phylogenetic uncertainty (see text). Panels on the right focus on the last 20 Ma, in which differences in regional assembly are most apparent. In the Hengduan Mountains region, cumulative \textit{in situ} speciation overtakes colonization about 8 Ma, whereas for the Himalayas-QTP, colonization remains the dominant process. \textit{In situ} speciation thus appears to have played a disproportionately large role in assembling the Hengduan Mountains flora since the late Miocene compared to the Himalayas-QTP, consistent with the theory of uplift-driven diversification in the Hengduan Mountains region.}
\label{fig:cumevents}
\end{figure}

\begin{figure}
\begin{center}
\includegraphics[width=.99\textwidth]{figures/figure_speciation_rates/figure_speciation_rates.pdf}
\end{center}
\caption{Rolling estimates of \textit{in situ} speciation rates through time for the Hengduan Mountains and Himalayas-QTP regions from inferred biogeographic histories of 18 plant clades. Lines indicate medians and shaded areas indicate 5--95\% quantile intervals from 500 pseudoreplicated joint histories designed to account for phylogenetic uncertainty (see text). Regional rates begin to diverge about 8 Ma, with the Hengduan Mountains showing a striking increase in \textit{in situ} speciation relative to the Himalayas-QTP.}
\label{fig:speciation}
\end{figure}

\begin{figure}
\begin{center}
\includegraphics[width=.99\textwidth]{figures/figure_dispersal_rates/figure_dispersal_rates.pdf}
\end{center}
\caption{Rolling estimates of colonization rates through time for the Hengduan Mountains, Himalayas-QTP, and temperate/boreal East Asia regions from inferred biogeographic histories of 18 plant clades. Lines indicate medians and shaded areas indicate 5–95\% quantile intervals from 500 pseudoreplicated joint histories designed to account for phylogenetic uncertainty (see text). Dispersal between the Hengduan Mountains and Himalayas-QTP increases in the last 2 Ma relative to dispersal between either region and temperate/boreal East Asia.}
\label{fig:dispersal}
\end{figure}

\begin{figure}
\begin{center}
\includegraphics[width=.99\textwidth]{figures/Rhododendron-supfig/Rhododendron-supfig.pdf}
\end{center}
\caption{Reconstructions of ancestral geographic range (left) and net diversification rate (right) on the maximum clade credibility tree, with branch lengths set to posterior means, for \textit{Rhododendron}. Ancestral ranges are maximum-likelihood estimates at the start and end of each branch. Net diversification values are branch-segment means of the posterior distribution estimated by BAMM. Filled circles on the right indicate branches that appear in the 95\% credible set of distinct shift configurations, with the size and label of a circle indicating the cumulative probability of the branch over all configurations in the credible set. On the left, the marginal odds ratio for a shift in diversification regime along a branch is drawn for branches where the ratio exceeds 20. Geographic regions are coded as follows: HEN, Hengduan Mountains; HIM, Himalayas-QTP; EAS, temperate-boreal East Asia; SEA, Southeast Asia; CWA, central/western Asia; EUR, Europe; IND, India; AFR, Africa; NAM, North America; SAM, South America; AUS, Australasia. Hengduan species cluster primarily in 2 clades, both of which show evidence of ancestral shifts to higher diversification rate.}
\label{fig:rhododendron}
\end{figure}

\end{document}

%%% Local Variables:
%%% mode: latex
%%% TeX-master: t
%%% End:
