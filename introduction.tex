\section{Introduction}

% Assembly of biodiversity hotspots - when, how, why? 2 basic processes: dispersal (immigration) and in situ diversification (speciation). Reconstructing these dynamics over time, in the context of geological and climatological events, is important to understanding the drivers/causes of major patterns in biogeography and macroecology (e.g., latitudinal diversity gradient).

Central to understanding global patterns of biodiversity are considerations of biotic assembly: for a given region, when and how did resident species accumulate? Of primary interest is tempo (the rate of accumulation) and mode (the process, whether by colonization via dispersal, or \textit{in situ} lineage diversification). We wish to know how and why these vary in time and space.

For mountains, well-known for harboring a disproportionate fraction of terrestrial species, a common hypothesis is that of ``uplift-driven diversification''---that orogeny creates conditions favoring \textit{in situ} speciation of resident lineages. Among global biodiversity hotspots, the mountain ranges surrounding the Qinghai-Tibetan Plateau (QTP) are unusual and enigmatic, in that they are neither tropical nor Mediterranean in climate, and despite increasing interest from biogeographers, their biotic assembly remains poorly understood \citep{Favre2015,Wen2014,Hughes2015}. The mountains form 3 distinct hotspots of biodiversity which respectively lie to the west, south, and east of the QTP's central high desert: the Central Asian mountains (Altai and Tianshan ranges), the Himalayas, and the Hengduan Mountains region \citep{Favre2015}. Of these, the richest in plant diversity is the Hengduan Mountains, with a vascular flora of about 12,000 species in an area of about 500,000 km$^2$ \citep{Boufford2014,LiEtLi1993,Wu1988}. In this study, we seek to better understand the origins of this remarkable flora through the lens of historical biogeography. In particular, was \textit{in situ} diversification accelerated by uplift of the Hengduan Mountains?

%The mountains actually form 2 distinct hotspots of biodiversity, roughly demarcated by the Tsangpo River Gorge in southeastern Tibet. The richer Hengduan Mountains lie to the east of this point, running north-south along the southeastern periphery of the QTP, with a flora of ca.\ 12,000 species of vascular plants in ca.\ 500,000 km$^2$ (\cite{Boufford2014},\cite{LiEtLi1993},\cite{Wu1988}); and the Himalayas lie to the west, running east-west along the QTP's southern edge, with ca.\ 10,000 species in ca.\ 750,000 km$^2$ (Conservation international, www.cepf.net/resources/hotspots/). However, because the floras share a substantial number of species (XX\%), they are commonly considered a single biogeographic region (e.g. \cite{Li1980},\cite{MengEtAl2008}).

The geological history of the QTP and its surrounding ranges is quite complex, and many details remain controversial, with different lines of evidence often yielding conflicting inferences. However, some points of consensus have emerged from recent syntheses \citep{WangC2014,Favre2015,Deng2015,Renner2016}. One is that the central plateau was uplifted first, forming a "proto-QTP" as early as 40 Mya, with subsequent outward extensions by the early Miocene \citep{Rowley2006,WangC2014}. By the late Miocene, 8--10 Mya, all the mountains surrounding the QTP to the south, west, and north had reached their current elevations \citep{Spicer2003,Fang2005,WangY2012,Deng2015}. By contrast, uplift of the Hengduan Mountains region, at the southeastern margin, is generally believed to have been rapid and recent, occurring mainly between the late Miocene and late Pliocene \citep{kirby2002,clark2005,WangE2012,Wang2014,Meng2016,SunB2011}.% during a time of global cooling,  intensification of the Asian monsoon, and increasing aridification of Central Asia \citep{AnZ2001}. [NOTE: An 2001 seems largely discredited at this point]

The Hengduan Mountains are thus younger than the rest of the QTP and relatively small in area, but conspicuously rich in species. If floristic assembly tracks orogeny, this suggests an elevated tempo since the late Miocene, compared to adjacent regions; but what of mode? Either colonization, \textit{in situ} diversification, or both processes must have been accelerated. In biogeographic studies of QTP-associated clades, the hypothesis of uplift-driven \textit{in situ} diversification is clearly favored over colonization, being commonly invoked to explain phylogenetic and phylogeographic divergences (e.g., \citealt{LiuJ2006,WangY2009,ZhangJ2014,GaoY2013}). However, conspicuously lacking from these studies are quantitative analyses that explicitly measure rates of diversification and colonization, and compare them across regions and time \citep{Wen2014,Favre2015}. Consequently, the idea that uplift-driven diversification has contributed disproportionately to floristic assembly in the Hengduan Mountains has yet to be rigorously tested.

% What historical biogeographic factors explain this pattern? Here we frame the question in terms of the two basic processes of regional lineage accumulation: \textit{in situ} diversification and colonization. The former is commonly thought to be accelerated by mountain uplift, because topographic complexity can increase the potential for speciation via allopatric isolation of populations, and adaptive divergence along environmental/elevational gradients \citep{Hughes2006,LiuJ2006,Xu2010,Hoorn2013,Hughes2015}. However, it is also possible that by creating and/or expanding habitats (e.g., of the alpine zone), uplift increases the potential for successful colonization of pre-adapted lineages dispersing from other regions \citep{Zhou2013,Lagomarsino2016}. In short, uplift might be expected to accelerate both diversification and colonization.



%The regions have often been considered a single floristic unit based on shared taxa (e.g., REFS).

%differ most conspicuously in their general orientation, with the Himalayas running east-west, and the Hengduan Mountains running north-south.

%Therefore, the nascent Hengduan region surely had close biogeographic connections to pre-existing alpine environments to the north and west, from which mountain-adapted clades could have dispersed. For example, in Cyananthus (Campanulaceae), phylogenetic analysis suggests that vicariance triggered by the uplift of the QTP accelerated the divergence of regional clades (sections), and that the Hengduan Mountains were colonized from the Himalayas {Zhou, 2013 #94}. More complex pattern has been found in the genus Anaphalis, (Asteraceae) that Anaphalis first radiated in the eastern Himalayas (Hengduan Mt) and migrated into eastern Asia, western Himalayas, as well as into North America, and SE Asia. This means Hengduan Mt mainly act as both species cradle and source of other floras. A general picture is still lacking and quantitative analyses are needed to address the contribution of immigration.

%None of these studies explicitly quantified diversification rates between HMH and other regions which makes hard to conclude whether extraordinary plant diversity in HMH is due to rapid diversification. Moreover, estimates of the time spans of these radiations vary widely (between 20-1.56 Ma), and in most cases are based only on secondary molecular-clock calibrations. Therefore, reconstructing the diversification rates dynamics across different clades will provide new insights into diversification processes in the HMH.

%It remains untested that how much immigration by pre-adapted species from adjacent regions had contributed to the exceptional plant diversity in the HMH. Much effort has been applied to reconstruct geographic patterns for clades in the QTP and adjacent regions (). Different hypotheses have been proposed in explaining current biogeographic pattern. Some studies indicate that most of the clades originated in the QTP and adjacent regions and started to specify in other Northern Hemisphere regions (e.g. Zhang et al., 2007, 2009; Xu et al., 2010; Zhang et al., 2014(Rhodiola)). Alternatively, some clades originated in other regions and radiated in the QTP (e.g. Liu et al., 2002; Tu et al., 2010). However, little is known when, from where and to which extent immigration has contributed to the spectacular diversity in the HMH in relative to in situ speciation. Were these biogeographic patterns associated with geological and environmental changes?

%"Uplift-driven diversification" hypothesis. Many recent phylogenetic studies of particular clades have invoked uplift of the QTP in general, and the Hengduan Mountains in particular, as a causal factor in lineage diversification. (see Favre et al 2015).

%To investigate the claim of "uplift-driven diversification", need to measure rates of in situ diversification and immigration of lineages over time. Also need to know when uplift occurred.

%If the Himalayas are indeed older than the Hengduan Mountains, we can speculate that the assembly of its flora began earlier, coinciding temporally with orogeny, and was dominated initially by a high rate of \textit{in situ} radiations of resident lineages (i.e., those already established on the southern edge of the QTP), or of early colonizing lineages from neighboring mountains, such as the Altai-Tienshan. Assuming density-dependent effects, the rate of assembly would have then tapered as the flora matured and the Himalayas approached their current height around 10 Mya. By contrast, as the Hengduan Mountains subsequently rose to the east, it seems plausible that the assembly of its flora was from the outset a more dynamic balance between colonization (primarily from the established flora of the geographically proximate and ecologically similar Himalayas) and \textit{in situ} diversification, and that the combined rates of these processes have been rising as orogeny has proceeded since the late Miocene.

In this study we use the evolutionary histories of multiple plant groups to study the floristic assembly of the Hengduan Mountains region, focusing on comparisons to adjacent regions, especially the Himalayas and other geologically older parts of the QTP. We infer regional rates of diversification and colonization through time from fossil-calibrated molecular chronograms and reconstructions of ancestral range and rates of lineage diversification, using data from 18 clades of vascular plants chosen for their potential to inform the biogeographic history of the Hengduan Mountains. Our primary aim is to discover the differences in tempo and mode of biotic assembly that help account for the remarkable diversity of the Hengduan flora.

%Methods of inferring biogeographic dynamics over evolutionary timescales. Fossil record - track the diversity and occurrence of species/clades in space directly through time. Alternatively, use time-calibrated phylogenies of extant species, and inferences of historical biogeography - where were lineages in the past, where did they diversify, and when/how did they move between regions. Both approaches: issue of sampling - incompleteness, bias. Phylogenies/anc. range reconstruction: issue of extinction.

%In this study we study the historical assembly of the Hengduan Mountains biodiversity hotspot using biogeographic inferences from multiple clades of seed plants. Our primary goal is to test the hypothesis that, compared to the Himalayas, its richer flora is the result of higher rates of \textit{in situ} diversification than immigration. A secondary hypothesis is that lineage-range dynamics in the Hengduan Mountains over time are qualitatively distinct from those in adjacent regions - that is, we wish to study the empirical justification for distinguishing it from the Himalayas. Multiple clades: time-calibrated phylogenies, species ranges. Ancestral range reconstruction yields inferences about when and how lineages moved and proliferated, integrated over uncertainty in topologies, node ages, and precise sequences of events.
%%% Local Variables:
%%% mode: latex
%%% TeX-master: "master"
%%% End:
