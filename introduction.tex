\section{Introduction}

% Assembly of biodiversity hotspots - when, how, why? 2 basic processes: dispersal (immigration) and in situ diversification (speciation). Reconstructing these dynamics over time, in the context of geological and climatological events, is important to understanding the drivers/causes of major patterns in biogeography and macroecology (e.g., latitudinal diversity gradient).

Understanding when and how regional biotas were assembled is central to understanding why biodiversity is distributed unevenly on Earth. Among global biodiversity hotspots---regions of unusually high species richness and endemism---the mountains along the southern edge of the Qinghai-Tibetan Plateau (QTP) are unusual in being neither tropical nor Mediterranean in climate, and they are also enigmatic, because despite increasing interest from biogeographers, the timing, tempo, and mode of their biotic assembly remains poorly understood.

The mountains actually form 2 distinct hotspots of biodiversity, roughly demarcated by the Tsangpo River Gorge in southeastern Tibet. The richer Hengduan Mountains lie to the east, with a flora of ca.\ 12,000 species of vascular plants in ca.\ 500,000 km$^2$ (Boufford; Li, 1993 #2; Wu, 1988 #1); and the Himalayas to the west, with ca.\ 10,000 species in ca.\ 750,000 km$^2$ (REF).

Any consideration of the biogeography of these floras necessarily begins with the formation of the mountains themselves and the associated changes in climate. Uplift of the QTP is intimately associated with development of the Asian monsoon and aridification of the regions north and west of the Himalayas (reviewed by Favre et al. 2014). Generally speaking, the orogenic history of the QTP and its surrounding mountains is extremely complex, and many details are controversial, with different lines of evidence often yielding conflicting inferences (see reviews in {Wang, 2014 #216} and Deng and Ding, 2016). Nevertheless, it seems likely that the central plateau was uplifted first, reaching its current elevation as early as 51-40 Ma to as late as 28 Ma (Xu, 2013 #217; Wang, 2014 #216), with subsequent outward growth. Along the southern edge of the plateau, the Himalayas rose earlier, reaching their current elevation no later than 11-9 Ma, while the Hengduan Mountains formed later, rising mainly after 10 Ma (Wang, 2012 #218; Mulch, 2006 #81; Sun, 2011 #43).

The Hengduan Mountains are thus younger than the Himalayas, but richer in species and smaller in area. What historical biogeographic factors explain this disparity? The question can be framed simply in terms of two basic processes: \textit{in situ} speciation and colonization. Speciation is commonly thought to be accelerated by mountain uplift, because topographic complexity increases the potential for allopatric isolation of populations and adaptive divergence along environmental/elevational gradients (REFS?). At a regional scale, this means we might expect higher rates of \textit{in situ} speciation during periods of uplift. However, it is also possible that by creating and/or expanding habitats (e.g., of the alpine zone), uplift also increases the rate of successful colonization via dispersal from other regions (REFS?).

Mountain uplift may thus increase rates of both \textit{in situ} speciation and colonization, but for the Hengduan Mountains, the former hypothesis of uplift-driven diversification is much more popular than the latter. In the rapidly growing systematics literature on taxa of the Hengduan Mountains and adjacent regions, uplift is commonly invoked as an explanation for phylogenetic or phylogeographic divergences (e.g., REFS). However, as noted in 2 recent reviews (Wen et al, Favre et al), these claims are rarely if ever backed up by quantitative analyses that (for example) measure rates of diversification and compare them across geographic regions. In other words, for the QTP in general and the Hengduan Mountains in particular, the uplift-driven diversification hypothesis has yet to be tested explicitly and rigorously.

%The regions have often been considered a single floristic unit based on shared taxa (e.g., REFS).

%differ most conspicuously in their general orientation, with the Himalayas running east-west, and the Hengduan Mountains running north-south.

%Therefore, the nascent Hengduan region surely had close biogeographic connections to pre-existing alpine environments to the north and west, from which mountain-adapted clades could have dispersed. For example, in Cyananthus (Campanulaceae), phylogenetic analysis suggests that vicariance triggered by the uplift of the QTP accelerated the divergence of regional clades (sections), and that the Hengduan Mountains were colonized from the Himalayas {Zhou, 2013 #94}. More complex pattern has been found in the genus Anaphalis, (Asteraceae) that Anaphalis first radiated in the eastern Himalayas (Hengduan Mt) and migrated into eastern Asia, western Himalayas, as well as into North America, and SE Asia. This means Hengduan Mt mainly act as both species cradle and source of other floras. A general picture is still lacking and quantitative analyses are needed to address the contribution of immigration.

%None of these studies explicitly quantified diversification rates between HMH and other regions which makes hard to conclude whether extraordinary plant diversity in HMH is due to rapid diversification. Moreover, estimates of the time spans of these radiations vary widely (between 20-1.56 Ma), and in most cases are based only on secondary molecular-clock calibrations. Therefore, reconstructing the diversification rates dynamics across different clades will provide new insights into diversification processes in the HMH.

%It remains untested that how much immigration by pre-adapted species from adjacent regions had contributed to the exceptional plant diversity in the HMH. Much effort has been applied to reconstruct geographic patterns for clades in the QTP and adjacent regions (). Different hypotheses have been proposed in explaining current biogeographic pattern. Some studies indicate that most of the clades originated in the QTP and adjacent regions and started to specify in other Northern Hemisphere regions (e.g. Zhang et al., 2007, 2009; Xu et al., 2010; Zhang et al., 2014(Rhodiola)). Alternatively, some clades originated in other regions and radiated in the QTP (e.g. Liu et al., 2002; Tu et al., 2010). However, little is known when, from where and to which extent immigration has contributed to the spectacular diversity in the HMH in relative to in situ speciation. Were these biogeographic patterns associated with geological and environmental changes?

%"Uplift-driven diversification" hypothesis. Many recent phylogenetic studies of particular clades have invoked uplift of the QTP in general, and the Hengduan Mountains in particular, as a causal factor in lineage diversification. (see Favre et al 2015).

%To investigate the claim of "uplift-driven diversification", need to measure rates of in situ diversification and immigration of lineages over time. Also need to know when uplift occurred.

If the Himalayas are indeed older than the Hengduan Mountains, we can speculate that the assembly of its flora began earlier. Assembly would have been driven initially by a high rate of \textit{in situ} speciation of resident lineages, that is, lineages already established on the southern edge of the QTP, or early colonizing lineages from the Altai-Tienshan; this diversification would have then tapered as the Himalayas approached their current height around 10 Mya. By contrast, as the Hengduan Mountains subsequently rose to the east, it seems plausible that its flora was initially assembled primarily by colonizing lineages from the Himalayas that subsequently diversified \textit{in situ}. The accumulation of more species in less time than the Himalayas means that rates of \textit{in situ} speciation, colonization, or both must have been higher for the Hengduan Mountains.

In this study we test these hypotheses using phylogenetic inference methods, combining fossil-calibrated molecular clocks for divergence-time estimation, historical biogeographic inferences of ancestral ranges and movements, and reconstructions of macroevolutionary birth-death dynamics to yield quantitative comparisons of \textit{in situ} diversification and colonization across regions, focusing on their rates through time and their cumulative effects to the present.







%Methods of inferring biogeographic dynamics over evolutionary timescales. Fossil record - track the diversity and occurrence of species/clades in space directly through time. Alternatively, use time-calibrated phylogenies of extant species, and inferences of historical biogeography - where were lineages in the past, where did they diversify, and when/how did they move between regions. Both approaches: issue of sampling - incompleteness, bias. Phylogenies/anc. range reconstruction: issue of extinction.

%In this study we study the historical assembly of the Hengduan Mountains biodiversity hotspot using biogeographic inferences from multiple clades of seed plants. Our primary goal is to test the hypothesis that, compared to the Himalayas, its richer flora is the result of higher rates of \textit{in situ} diversification than immigration. A secondary hypothesis is that lineage-range dynamics in the Hengduan Mountains over time are qualitatively distinct from those in adjacent regions - that is, we wish to study the empirical justification for distinguishing it from the Himalayas. Multiple clades: time-calibrated phylogenies, species ranges. Ancestral range reconstruction yields inferences about when and how lineages moved and proliferated, integrated over uncertainty in topologies, node ages, and precise sequences of events.