\section{Introduction}

% Assembly of biodiversity hotspots - when, how, why? 2 basic processes: dispersal (immigration) and in situ diversification (speciation). Reconstructing these dynamics over time, in the context of geological and climatological events, is important to understanding the drivers/causes of major patterns in biogeography and macroecology (e.g., latitudinal diversity gradient).

Understanding when and how regional biotas were assembled is central to understanding why biodiversity is distributed unevenly on Earth. Among global biodiversity hotspots---regions of unusually high species richness and endemism---the mountains along the southern edge of the Qinghai-Tibetan Plateau (QTP) are both unusual and enigmatic: unusual because they constitute the richest hotspot that is neither tropical nor Mediterranean in climate, and enigmatic because, despite increasing interest from biogeographers, the timing, tempo, and mode of its biotic assembly remains poorly understood.

The hotspot can be broadly subdivided into two distinct regions, roughly demarcated by the Tsangpo Gorge in southeastern Tibet: the richer Hengduan Mountains lie to the east, with a flora of ca.\ 12,000 species in ca.\ 500,000 km$^2$ (Boufford; Li, 1993 #2; Wu, 1988 #1); and the Himalayas to the west, with ca.\ 10,000 species in ca.\ 750,000 km$^2$ (REF). They differ most conspicuously in their general orientation, with the Himalayas running east-west, and the Hengduan Mountains running north-south.

Any investigation of the assembly of these floras necessarily begins with the formation of the mountains themselves. Initiated by the collision of India with Eurasia in the early Eocene, the orogenic history of the QTP and adjacent regions is extremely complex, and the details are controversial, with different proxies yielding conflicting inferences (see reviews in {Wang, 2014 #216} and Deng and Ding, 2016). However, it seems likely that the central QTP was uplifted first, reaching its current elevation as early as 51-40 Ma or as late as 28 Ma {Xu, 2013 #217}{Wang, 2014 #216}, with subsequent growth of the plateau from the inside out (REF?). Along the southern edge, the Himalayan region was uplifted first, likely reaching its current elevation no later than 11-9 Ma, while the Hengduan Mountains are younger, forming mainly after 10 Ma {Wang, 2012 #218}{Mulch, 2006 #81}{Sun, 2011 #43}.

Something about monsoon development here

%Therefore, the nascent Hengduan region surely had close biogeographic connections to pre-existing alpine environments to the north and west, from which mountain-adapted clades could have dispersed. For example, in Cyananthus (Campanulaceae), phylogenetic analysis suggests that vicariance triggered by the uplift of the QTP accelerated the divergence of regional clades (sections), and that the Hengduan Mountains were colonized from the Himalayas {Zhou, 2013 #94}. More complex pattern has been found in the genus Anaphalis, (Asteraceae) that Anaphalis first radiated in the eastern Himalayas (Hengduan Mt) and migrated into eastern Asia, western Himalayas, as well as into North America, and SE Asia. This means Hengduan Mt mainly act as both species cradle and source of other floras. A general picture is still lacking and quantitative analyses are needed to address the contribution of immigration.

None of these studies explicitly quantified diversification rates between HMH and other regions which makes hard to conclude whether extraordinary plant diversity in HMH is due to rapid diversification. Moreover, estimates of the time spans of these radiations vary widely (between 20-1.56 Ma), and in most cases are based only on secondary molecular-clock calibrations. Therefore, reconstructing the diversification rates dynamics across different clades will provide new insights into diversification processes in the HMH.


It remains untested that how much immigration by pre-adapted species from adjacent regions had contributed to the exceptional plant diversity in the HMH. Much effort has been applied to reconstruct geographic patterns for clades in the QTP and adjacent regions (). Different hypotheses have been proposed in explaining current biogeographic pattern. Some studies indicate that most of the clades originated in the QTP and adjacent regions and started to specify in other Northern Hemisphere regions (e.g. Zhang et al., 2007, 2009; Xu et al., 2010; Zhang et al., 2014(Rhodiola)). Alternatively, some clades originated in other regions and radiated in the QTP (e.g. Liu et al., 2002; Tu et al., 2010). However, little is known when, from where and to which extent immigration has contributed to the spectacular diversity in the HMH in relative to in situ speciation. Were these biogeographic patterns associated with geological and environmental changes?

"Uplift-driven diversification" hypothesis: The expectation is that uplift has a general effect on clades - that it increases the potential for range subdivision/isolation, as well as adaptive divergence/radiation along environmental gradients associated with mountain systems. General prediction: rates of in situ diversification are higher during periods of uplift. However, also possible that uplift, in exposing/creating new habitat (e.g., alpine), increases the opportunity for colonization/immigration. A priori, we might expect the buildup/assembly of a mountain flora to be driven by both processes.

To investigate the claim of "uplift-driven diversification", need to measure rates of in situ diversification and immigration of lineages over time. Also need to know when uplift occurred.

The question can be framed as a comparison of the Hengduan Mountains with adjacent regions. Is there a general tendency for clades to diversify faster in the Hengduan Mountains? To what extent is the accumulation of species through time driven by \textit{in situ} diversification versus immigration (source-sink dynamics)?

The principal comparison is to the Himalayas. Conventional wisdom is that the Himalayas were uplifted earlier, and are hence older, than the Hengduan Mountains. In the Himalayas we might thus expect clades to have generally deeper histories of in situ diversification, and to have acted as a source of lineages for Hengduan Mountains colonization.

Methods of inferring biogeographic dynamics over evolutionary timescales. Fossil record - track the diversity and occurrence of species/clades in space directly through time. Alternatively, use time-calibrated phylogenies of extant species, and inferences of historical biogeography - where were lineages in the past, where did they diversify, and when/how did they move between regions. Both approaches: issue of sampling - incompleteness, bias. Phylogenies/anc. range reconstruction: issue of extinction.

In this study we study the historical assembly of the Hengduan Mountains biodiversity hotspot using biogeographic inferences from multiple clades of vascular plants. Our primary goal is to test the hypothesis that its species-rich flora is the result of higher rates of situ diversification than immigration. A secondary hypothesis is that lineage-range dynamics in the Hengduan Mountains over time are qualitatively distinct from those in adjacent regions - that is, we wish to study the empirical justification for distinguishing it from the Himalayas. Multiple clades: time-calibrated phylogenies, species ranges. Ancestral range reconstruction yields inferences about when and how lineages moved and proliferated, integrated over uncertainty in topologies, node ages, and precise sequences of events.