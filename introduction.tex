\section*{Introduction}

Assembly of biodiversity hotspots - when, how, why? 2 basic processes: dispersal (immigration) and in situ diversification (speciation). Time axis: museum vs. cradle. Reconstructing these dynamics over time, in the context of geological and climatological events, is important to understanding the drivers/causes of major patterns in biogeography and macroecology (e.g., latitudinal diversity gradient).

Hengduan Mountains: temperate biodiversity hotspot situated on the southeastern edge of the Tibetan Plateau. Unusual in not being tropical or having a Mediterranean climate. Adjacent to other hotspots: the Himalayas to the west (including the southern edge of the Qinghai-Tibetan Plateau), and the Gaoligong Mountains to the south. Together these regions are often considered a single biogeographic unit (...)

Vascular plant diversity in the Hengduan Mountains is very high - this is often attributed to accelerated speciation driven by uplift (many studies). However, this claim has never been investigated quantitatively: rates of in-situ diversification not measured, compared across adjacent regions, or contrasted with rates of lineage immigration over time. Also, most/all of these claims have been made in the context of single-clade analyses.

The question can be framed as a comparison of the Hengduan Mountains with adjacent regions. Is there a general tendency for clades to diversify faster in the Hengduan Mountains? To what extent is the accumulation of species through time driven by \textit{in situ} diversification versus immigration (source-sink dynamics)?

Methods of inferring biogeographic dynamics over evolutionary timescales. Fossil record - track the diversity and occurrence of species/clades in space directly through time. Alternatively, use time-calibrated phylogenies of extant species, and inferences of historical biogeography - where were lineages in the past, where did they diversify, and when/how did they move between regions. Both approaches: issue of sampling - incompleteness, bias. Phylogenies/anc. range reconstruction: issue of extinction.

In this study we study the historical assembly of the Hengduan Mountains biodiversity hotspot using biogeographic inferences from multiple clades of vascular plants. Our primary goal is to test the hypothesis that its species-rich flora is the result of higher rates of situ diversification than immigration. A secondary hypothesis is that lineage-range dynamics in the Hengduan Mountains over time are qualitatively distinct from those in adjacent regions - that is, we wish to study the empirical justification for distinguishing it from the Himalayas. Multiple clades: time-calibrated phylogenies, species ranges. Ancestral range reconstruction yields inferences about when and how lineages moved and proliferated, integrated over uncertainty in topologies, node ages, and precise sequences of events.