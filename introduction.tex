\section{Introduction}

Assembly of biodiversity hotspots - when, how, why? 2 basic processes: dispersal (immigration) and in situ diversification (speciation). Time axis: museum vs. cradle. Reconstructing these dynamics over time, in the context of geological and climatological events, is important to understanding the drivers/causes of major patterns in biogeography and macroecology (e.g., latitudinal diversity gradient).

Methods of inferring biogeographic dynamics over evolutionary timescales. Fossil record - track the diversity and occurrence of species/clades in space directly through time. Alternatively, use time-calibrated phylogenies of extant species, and inferences of historical biogeography - ancestral ranges, the timing and direction of movement between regions. Both approaches: issue of sampling - incompleteness, bias. Phylogenies/anc. range reconstruction: issue of extinction.

Hengduan Mountains: temperate biodiversity hotspot situated on the southeastern edge of the Tibetan Plateau. Unusual in not being tropical or having a Mediterranean climate. Adjacent to other hotspots: the Himalayas to the west (including the southern edge of the Qinghai-Tibetan Plateau), and the Gaoligong Mountains to the south. Together these regions are often considered a single biogeographic unit (...)

Vascular plant diversity in the Hengduan Mountains is very high - this is often attributed to accelerated speciation driven by uplift (many studies). However, rates of in-situ diversification not quantified, compared across adjacent regions, or contrasted with rates of lineage immigration over time.

In this study we address the question: how was the Hengduan Mountains flora assembled - by immigration or in situ diversification? How have these dynamics changed over time?