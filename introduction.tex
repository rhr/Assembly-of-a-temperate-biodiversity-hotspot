\section{Introduction}

Beginning in the early Eocene, 50$\pm$5 million years ago (Mya), the collision between Eurasian and Indian plates starting from 50 ± 5 Ma and post-collinsonal convergence has created the highest and largest plateau as well as extensive mountainous regions at the border. The mountains of the Southwest China hotspot, usually called as the Hengduan Mountain Hotspot (HMH), is located at the Hengduan Mountain regions, the southeastern edge of the QTP. This temperate-zone hotspot of biodiversity has a rich flora with ca. 8000-12000 species of seed plants in which ca. 3500 are endemic in a relatively small area, 364,000 km2 {Li, 1993 #2}{Wu, 1988 #1}. Compared to relatively well-known mountain radiations in the Andes of South America {Hughes, 2006 #71}{Hughes, 2015 #208}{Hoorn, 2010 #138} and the European Alpine system {Roquet, 2013 #116}, the causes of extraordinary diversity in HMH remains poorly understood.

Assembly of biodiversity hotspots - when, how, why? 2 basic processes: dispersal (immigration) and in situ diversification (speciation). Time axis: museum vs. cradle. Reconstructing these dynamics over time, in the context of geological and climatological events, is important to understanding the drivers/causes of major patterns in biogeography and macroecology (e.g., latitudinal diversity gradient).

Hengduan Mountains: temperate biodiversity hotspot situated on the southeastern edge of the Tibetan Plateau. Unusual in not being tropical or having a Mediterranean climate. Adjacent to other hotspots: the Himalayas to the west (including the southern edge of the Qinghai-Tibetan Plateau), and the Gaoligong Mountains to the south. Together these regions are often considered a single biogeographic unit (...). Vascular plant diversity is high in the Hengduan Mountains - measured per unit area, much higher than adjacent regions (the Himalayas or non-tropical East Asia). (Tropical levels of diversity? Can compare to Kalimantan.)

Most of studies consider the QTP as a whole, and none of these studies explicitly tested for differences in diversification rates between HMH and other regions which makes hard to conclude whether extraordinary plant diversity in HMH is due to rapid diversification. Moreover, estimates of the time spans of these radiations vary widely (between 20-1.56 Ma), and in most cases are based only on secondary molecular-clock calibrations. Therefore, reconstructing the diversification rates dynamics across different clades will provide new insights into diversification processes in the HMH.

High diversity of vascular plants in the Hengduan Mountains has often been attributed to accelerated speciation driven by uplift (many studies). This claim is generally made in the context of single-clade analyses (e.g., ...), but it has received little quantitative analysis.

It remains untested that how much immigration by pre-adapted species from adjacent regions had contributed to the exceptional plant diversity in the HMH. Much effort has been applied to reconstruct geographic patterns for clades in the QTP and adjacent regions (). Different hypotheses have been proposed in explaining current biogeographic pattern. Some studies indicate that most of the clades originated in the QTP and adjacent regions and started to specify in other Northern Hemisphere regions (e.g. Zhang et al., 2007, 2009; Xu et al., 2010; Zhang et al., 2014(Rhodiola)). Alternatively, some clades originated in other regions and radiated in the QTP (e.g. Liu et al., 2002; Tu et al., 2010). However, little is known when, from where and to which extent immigration has contributed to the spectacular diversity in the HMH in relative to in situ speciation. Were these biogeographic patterns associated with geological and environmental changes?

Though the uplift history of QTP and adjacent regions is extremely complex and detailed uplift processes are still controversial (see review in {Wang, 2014 #216} and Deng and Ding, 2016). Geological evidence showed that the central QTP reached it current elevation probably as early as 51-40 Ma, at least by 28 Ma {Xu, 2013 #217}{Wang, 2014 #216} followed by an inside-out growth pattern. The Himalayan region may have reached its current elevation at round 11-9 Ma. Recent evidence indicated that the uplift of the Hengduan Mt was mainly after 10 Ma {Wang, 2012 #218}{Mulch, 2006 #81}{Sun, 2011 #43}. Therefore, the nascent Hengduan region surely had close biogeographic connections to pre-existing alpine environments to the north and west, from which mountain-adapted clades could have dispersed. For example, in Cyananthus (Campanulaceae), phylogenetic analysis suggests that vicariance triggered by the uplift of the QTP accelerated the divergence of regional clades (sections), and that the Hengduan Mountains were colonized from the Himalayas {Zhou, 2013 #94}. More complex pattern has been found in the genus Anaphalis, (Asteraceae) that Anaphalis first radiated in the eastern Himalayas (Hengduan Mt) and migrated into eastern Asia, western Himalayas, as well as into North America, and SE Asia. This means Hengduan Mt mainly act as both species cradle and source of other floras. A general picture is still lacking and quantitative analyses are needed to address the contribution of immigration.

"Uplift-driven diversification" hypothesis: The expectation is that uplift has a general effect on clades - that it increases the potential for range subdivision/isolation, as well as adaptive divergence/radiation along environmental gradients associated with mountain systems. General prediction: rates of in situ diversification are higher during periods of uplift. However, also possible that uplift, in exposing/creating new habitat (e.g., alpine), increases the opportunity for colonization/immigration. A priori, we might expect the buildup/assembly of a mountain flora to be driven by both processes.

To investigate the claim of "uplift-driven diversification", need to measure rates of in situ diversification and immigration of lineages over time. Also need to know when uplift occurred.

The question can be framed as a comparison of the Hengduan Mountains with adjacent regions. Is there a general tendency for clades to diversify faster in the Hengduan Mountains? To what extent is the accumulation of species through time driven by \textit{in situ} diversification versus immigration (source-sink dynamics)?

The principal comparison is to the Himalayas. Conventional wisdom is that the Himalayas were uplifted earlier, and are hence older, than the Hengduan Mountains. In the Himalayas we might thus expect clades to have generally deeper histories of in situ diversification, and to have acted as a source of lineages for Hengduan Mountains colonization.

Methods of inferring biogeographic dynamics over evolutionary timescales. Fossil record - track the diversity and occurrence of species/clades in space directly through time. Alternatively, use time-calibrated phylogenies of extant species, and inferences of historical biogeography - where were lineages in the past, where did they diversify, and when/how did they move between regions. Both approaches: issue of sampling - incompleteness, bias. Phylogenies/anc. range reconstruction: issue of extinction.

In this study we study the historical assembly of the Hengduan Mountains biodiversity hotspot using biogeographic inferences from multiple clades of vascular plants. Our primary goal is to test the hypothesis that its species-rich flora is the result of higher rates of situ diversification than immigration. A secondary hypothesis is that lineage-range dynamics in the Hengduan Mountains over time are qualitatively distinct from those in adjacent regions - that is, we wish to study the empirical justification for distinguishing it from the Himalayas. Multiple clades: time-calibrated phylogenies, species ranges. Ancestral range reconstruction yields inferences about when and how lineages moved and proliferated, integrated over uncertainty in topologies, node ages, and precise sequences of events.