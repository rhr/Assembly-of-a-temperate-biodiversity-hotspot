\begin{figure}
\centering
\includegraphics[width=.99\linewidth]{figures/regions.pdf}
\caption{Map of the 11 geographic regions used for ancestral range
  analyses in Lagrange.  HEN = Hengduan Mountains, HIM =
  Himalayas-QTP, EAS = temperate/boreal East Asia, SEA = Southeast
  Asia, CWA = Central/Western Asia, EUR = Europe, IND = India, AFR =
  Africa, NAM = North America, SAM = South America, AUS =
  Australasia. Lines and common borders indicate dispersal routes
  allowed in Lagrange.}
\label{fig:regions}
\end{figure}


\begin{figure}
  \caption{Reconstructions of ancestral geographic range (left) and
    net diversification rate (right) on the maximum clade credibility
    tree, with branch lengths set to posterior means, for each ingroup
    taxon (excluding \textit{Rhododendron}, shown as
    Fig.~4). Ancestral ranges are maximum-likelihood estimates at the
    start and end of each branch. Net diversification values are
    branch-segment means of the posterior distribution estimated by
    BAMM. Filled circles on the right indicate branches that appear in
    the 95\% credible set of distinct shift configurations, with the
    size and label of a circle indicating the cumulative probability
    of the branch over all configurations in the credible set. On the
    left, the marginal odds ratio for a shift in diversification
    regime along a branch is drawn for branches where the ratio
    exceeds 20. Geographic regions are coded as follows: HEN, Hengduan
    Mountains; HIM, Himalayas-QTP; EAS, temperate-boreal East Asia;
    SEA, Southeast Asia; CWA, central/western Asia; EUR, Europe; IND,
    India; AFR, Africa; NAM, North America; SAM, South America; AUS,
    Australasia.}
  \label{fig:ancranges}
\end{figure}

\addtocounter{figure}{-1}

\begin{figure}
\begin{subfigure}{\textwidth}
\centering
\includegraphics[width=.99\linewidth]{figures/Acer-supfig.pdf}
\label{fig:acer}
\caption{\textit{Acer}}
\end{subfigure}
\end{figure}

\begin{figure}
  \ContinuedFloat
\begin{subfigure}{\textwidth}
\centering
\includegraphics[width=.99\linewidth]{figures/Allium-supfig.pdf}
\label{fig:allium}
\caption{\textit{Allium}}
\end{subfigure}
\end{figure}

\begin{figure}
  \ContinuedFloat
\begin{subfigure}{\textwidth}
\centering
\includegraphics[width=.99\linewidth]{figures/Clematis-supfig.pdf}
\label{fig:allium}
\caption{Clematidinae+Anemoninae}
\end{subfigure}
\end{figure}

\begin{figure}
  \ContinuedFloat
\begin{subfigure}{\textwidth}
\centering
\includegraphics[width=.99\linewidth]{figures/Delphineae-supfig.pdf}
\label{fig:allium}
\caption{Delphineae}
\end{subfigure}
\end{figure}

\begin{figure}
  \ContinuedFloat
\begin{subfigure}{\textwidth}
\centering
\includegraphics[width=.99\linewidth]{figures/Cyananthus-supfig.pdf}
\label{fig:allium}
\caption{\textit{Cyananthus}}
\end{subfigure}
\end{figure}

\begin{figure}
  \ContinuedFloat
\begin{subfigure}{\textwidth}
\centering
\includegraphics[width=.99\linewidth]{figures/Isodon-supfig.pdf}
\label{fig:allium}
\caption{\textit{Isodon}}
\end{subfigure}
\end{figure}

\begin{figure}
  \ContinuedFloat
\begin{subfigure}{\textwidth}
\centering
\includegraphics[width=.99\linewidth]{figures/Ligularia-supfig.pdf}
\label{fig:allium}
\caption{\textit{Ligularia-Cremanthodium-Parasenecio} complex}
\end{subfigure}
\end{figure}

\begin{figure}
  \ContinuedFloat
\begin{subfigure}{\textwidth}
\centering
\includegraphics[width=.99\linewidth]{figures/Meconopsis-supfig.pdf}
\label{fig:allium}
\caption{\textit{Meconopsis}}
\end{subfigure}
\end{figure}

\begin{figure}
  \ContinuedFloat
\begin{subfigure}{\textwidth}
\centering
\includegraphics[width=.99\linewidth]{figures/Microsoroideae-supfig.pdf}
\label{fig:allium}
\caption{Microsoroideae}
\end{subfigure}
\end{figure}

\begin{figure}
  \ContinuedFloat
\begin{subfigure}{\textwidth}
\centering
\includegraphics[width=.99\linewidth]{figures/Pinaceae-supfig.pdf}
\label{fig:allium}
\caption{Pinaceae}
\end{subfigure}
\end{figure}

\begin{figure}
  \ContinuedFloat
\begin{subfigure}{\textwidth}
\centering
\includegraphics[width=.99\linewidth]{figures/Polygoneae-supfig.pdf}
\label{fig:allium}
\caption{Polygoneae}
\end{subfigure}
\end{figure}

\begin{figure}
  \ContinuedFloat
\begin{subfigure}{\textwidth}
\centering
\includegraphics[width=.99\linewidth]{figures/Primulaceae-supfig.pdf}
\label{fig:allium}
\caption{Primulaceae}
\end{subfigure}
\end{figure}

\begin{figure}
  \ContinuedFloat
\begin{subfigure}{\textwidth}
\centering
\includegraphics[width=.99\linewidth]{figures/Rhodiola-supfig.pdf}
\label{fig:allium}
\caption{\textit{Rhodiola}}
\end{subfigure}
\end{figure}

\begin{figure}
  \ContinuedFloat
\begin{subfigure}{\textwidth}
\centering
\includegraphics[width=.99\linewidth]{figures/Rosa-supfig.pdf}
\label{fig:allium}
\caption{\textit{Rosa}}
\end{subfigure}
\end{figure}

\begin{figure}
  \ContinuedFloat
\begin{subfigure}{\textwidth}
\centering
\includegraphics[width=.99\linewidth]{figures/Saussurea-supfig.pdf}
\label{fig:allium}
\caption{\textit{Saussurea}}
\end{subfigure}
\end{figure}

\begin{figure}
  \ContinuedFloat
\begin{subfigure}{\textwidth}
\centering
\includegraphics[width=.99\linewidth]{figures/Saxifragaceae-supfig.pdf}
\label{fig:allium}
\caption{Saxifragaceae-Grossulariaceae}
\end{subfigure}
\end{figure}

\begin{figure}
  \ContinuedFloat
\begin{subfigure}{\textwidth}
\centering
\includegraphics[width=.99\linewidth]{figures/Thalictrum-supfig.pdf}
\label{fig:allium}
\caption{\textit{Thalictrum}}
\end{subfigure}
\end{figure}

%%% Local Variables:
%%% mode: latex
%%% TeX-master: "SI"
%%% End:
