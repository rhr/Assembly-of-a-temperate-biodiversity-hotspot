\subsection{Results}

We obtained time-calibrated molecular phylogenies for 18 clades, sampling a total of 3,266 ingroup species of which 899 occur in the Hengduan Mountains, 628 in the Himalayas, and 1,165 in temperate East Asia. The proportion of species sampled was roughly the same across these regions, in the range of 50--60\% (SI REF).

Given the geological view that the Hengduan Mountains are younger than the Himalayas, we expected to find evidence that its flora was assembled more recently. Instead, our joint biogeographic reconstructions of clades show surprisingly deep phylogenetic histories of Hengduan occupancy, with initial floristic assembly of the Hengduan Mountains pre-dating that of the Himalayas. Colonization of the Hengduan Mountains began 96--59 Mya and \textit{in situ} speciation began 80--34 Mya (95\% quantiles of 500 pseudoreplicates), while for the Himalayas, the corresponding times are XX--XX and XX--XX Mya. For both regions, floristic assembly is thus dominated initially by colonization, with a delayed onset of in situ speciation; both processes occur at more or less constant rates, with the rate of speciation exceeding that of colonization. In the Hengduan Mountains, the cumulative contribution of \textit{in situ} speciation to regional diversity overtakes that of colonization around 15 Mya, after which both processes 

; by contrast, for temperate East Asia, it is dominated by \textit{in situ} speciation. 